% !TEX root = ../thesis.tex

\chapter{Záver}
\label{summary}
Cieľom našej práce bolo uvedenie čitateľa do problematiky VPN sietí, so zameraním na použitie kryptografického algoritmu XOODOO permutácie za účelom zabezpečenia komunikácie. V~prvej časti práci sme zadefinovali potrebné pojmy spojené s problematikou VPN sietí. Následne sme postupným opisom charakterizovali čo sa deje v~sieťach typu VPN. Na základe naštudovanej literatúry sme vytvorili klasifikácie VPN sietí. Rozdelenie bolo realizované podľa logickej topológie a dát vstupujúcich do~šifrovania. Pomocou nadobudnutých informácií sme charakterizovali a špecifikovali činnosti implementované v známych VPN protokoloch. V závere prvej kapitoly sme zhrnuli rozdelenie VPN sietí pomocou grafu.

Druhá kapitola sa venuje problematike ľahkej kryptografie. V nej sme charakterizovali tento pomerné nový odbor v oblasti kryptografie. Následne sme pokračovali s opisom XOODOO permutácie, ktorú zaraďujeme do tejto kategórie. XOODOO permutácia je súčasťou kryptografického balíka Xoodyak, ktorý je jedným z finalistov štandardizačného procesu NIST v ľahkej kryptografie. Opísali sme činnosti v algoritme a vysvetlili možnosti použitia. 

Tretia kapitola obsahuje postupy prípravy prostredí na experimentovanie. Konfiguráciu virtualizovaných OS a opis jednotlivých nástrojov, ktoré boli pri práci použité. Opísali sme Wintun adaptér na tvorbu virtuálnych tunelovacích rozhraní v OS Windows. Za účelom použitia pri výučbe v špecializovaných predmetoch orientovaných na bezpečnost v počítačových sieťach, sme vytvorili jednoduché demo, ktoré demonštruje odosielanie a prijímanie ICMP paketov pomocou tohto adaptéru.

Štvrtá kapitola sa zameriava na praktickú demonštráciu jednoduchej VPN siete s využitím XOODOO permutácie. V úvode sme charakterizovali program DSVPN. Následne sme praktickým experimentom overili jeho funkčnosť. Z tejto činnosti sme vypracovali článok do Zborníka vedeckých prác TUKE FEI \cite{clanok}. Podrobne sme analyzovali zdrojový kód DSVPN. Popri analýze sme názorným spôsobom ukázali čo sa deje s paketmi. Experimentálne sme zmerali rýchlosť a počet cyklov implementácie XOODOO permutácie v DSVPN. Súčasťou našej práce bolo aj vytvorenie kompatibility DSVPN o OS Windows a opis vykonaných zmien. 

V piatej kapitole sme zhrnuli dosiahnuté výsledky z praktickej časti práce. Celý obsah prílohy A vrátane samotnej práce sme uverejnili na git stránku Github pod profilom mr171hg\footnote{\url{https://github.com/mr171hg/DiplomaProject}}. Dôvodom je dostupnosť týchto informácií pre každého, kto sa potrebuje v problematike VPN zorientovať.  

Výsledná podoba práce spoločne s demonštratívnymi príkladmi zdrojových kódov a ostatnými prílohami boli spracované tak, aby bolo možné ich použitie v špecializovaných predmetoch zameraných na bezpečnosť v počítačových sieťach.

Pre účely rozšírenia práce by sme navrhovali pokračovať v práci na kompatibilite DSVPN. Rozšíriť implementáciu o viacero kryptografických algoritmov a následne experimentálne merať. Aplikovať program DSVPN na sieťové zariadenie a vyskúšať reálnu prevádzku. Refaktorovať a optimalizovať zdrojový kód. Rozšíriť program o možnosti pripojenia viacerých klientov na jeden server s využitím paralelného programovania s viac vláknami. 
