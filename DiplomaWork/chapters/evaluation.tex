% !TEX root = ../thesis.tex
\chapter{Vyhodnotenie dosiahnutých výsledkov}
V rámci tejto kapitoly zhrnieme dosiahnuté výsledky z analýzy, experimentálnych meraní a rozšírenia kompatibility pre OS Windows.
\section{Analýza jednoduchej VPN siete}
V úvode praktickej časti sme postupnou analýzou prešli jednotlivé časti zdrojového kódu DSVPN. Názorne sme opisovali dôležité bloky a funkcionalitu implementovanú v tejto jednoduchej VPN. Na základe nadobudnutých znalostí sme vypracovali stavové diagramy, ktoré čitateľovi jednoznačne vysvetľujú čo sa v programe realizuje. Na základe týchto informácií, dokáže pochopiť ako sa v praxi realizuje VPN sieťové spojenie.
\section{Experimentálne meranie autentizovaného šifrovania pomocou permutácie XOODOO}
Popri analýze zdrojového kódu sme experimentálne overili funkcionalitu autentizovaného šifrovania pomocou permutácie XOODOO v jednoduchej VPN sieti. Z výsledkov experimentálnych meraní implementácia XOODOO algoritmu, môžeme zhodnotiť, že autentizované šifrovanie a dešifrovanie dokáže držať na úrovní 40-70 mikrosekúnd. Čas vykonávania bol priamo úmerný veľkosti vstupujúcich dát a prostredia, na ktorom aplikácia beží. Na základe meraní môžeme tvrdiť, že sa jedná o extrémne rýchlu a výpočtovo nenáročnú implementáciu. Rád by som však oboznámil čitateľa s faktom, že do meraní nám vo veľkej miere zasahoval OS. Simuláciu bežného prostredia sme uskutočnili za pomoci 2 virtuálnych OS a~jedného natívneho OS, na ktorom virtuálizácia prebiehala. Virtualizované zariadenia mali obdobne aj obmedzený výpočtový výkon a pridelené prostriedky. To sa prejavovalo v pomalších odozvách systémov na požiadavky používateľa. Na základe uvedených faktov konštatujeme, že výsledná rýchlosť na bežnom používateľskom prostredí by mala byť ešte rýchlejšia.

\section{Výsledky dosiahnuté pri tvorbe kompatibility \\DSVPN pre OS Windows}
Jedným z cieľov našej práce bolo aj rozšírenie pôvodného riešenia DSVPN o~kompatibilitu pre OS Windows. Samozrejmosťou bolo zachovanie pôvodnej kompatibility v rámci ostatných OS. V princípe sa nám podarilo vytvoriť tunelové spojenie medzi VPN klientom a VPN serverom so zabezpečeným prenosom dát. Taktiež je možné nadviazať spojenie nie len medzi zariadeniami s OS Windows, ale vieme spojiť ľubovoľné OS z množiny podporovaných DSVPN.  

Problém riešenia nastáva pri smerovaní paketov naspäť k pôvodnému vlastníkovi. Pomocou sieťových príkazov a monitoringu premávky sa nám nepodarilo doteraz zistiť, prečo sa odpoveď na pôvodnú požiadavku nedokáže vrátiť na~zariadenie. Tento úkon sa deje aj napriek tomu, že pri monitoringu vidíme ako paket s odpoveďou opúšťa zariadenie a prichádza na to, ktoré má dostať odpoveď. Súčasná hypotéza je, že nám uniká niečo pri smerovaní paketu z pozície OS Windows. Tento problém sa budeme snažiť ešte v blízkej budúcnosti vyriešiť, nakoľko by sa nemalo jednať o veľký problém. Čitateľ bude mať prístup ku aktuálnemu kódu pomocou stránky github.   