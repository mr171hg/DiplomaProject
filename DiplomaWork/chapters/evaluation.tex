% !TEX root = ../thesis.tex
\chapter{Vyhodnotenie dosiahnutých výsledkov}
V rámci tejto kapitoly zhrnieme dosiahnuté výsledky z analýzy, experimentálnych meraní a rozšírenia kompatibility pre OS Windows.
\section{Analýza jednoduchej VPN siete}
V úvode praktickej časti sme postupnou analýzou prešli jednotlivé časti zdrojového kódu DSVPN. Názorne sme opisovali dôležité bloky a funkcionalitu implementovanú v tejto jednoduchej VPN. Na základe nadobudnutých znalostí sme vypracovali stavové diagramy, ktoré čitateľovi jednoznačne vysvetľujú čo sa v~programe realizuje. Na základe týchto informácií, dokáže pochopiť ako sa v~praxi realizuje VPN sieťové spojenie.
\section{Experimentálne meranie autentizovaného šifrovania pomocou permutácie XOODOO}
Popri analýze zdrojového kódu sme experimentálne overili funkcionalitu autentizovaného šifrovania pomocou permutácie XOODOO v jednoduchej VPN sieti.  Simuláciu bežného prostredia sme uskutočnili za pomoci 2 virtuálnych OS a~jedného natívneho OS, na ktorom virtuálizácia prebiehala. Virtualizované zariadenia mali obdobne aj obmedzený výpočtový výkon a pridelené prostriedky. To sa prejavovalo v pomalších odozvách systémov na požiadavky používateľa. 

Zo získaných výsledkov v tabuľke \ref{tabmer} sme usúdili, že virtuálne prostredie nie je vhodné pre meranie. Z toho dôvodu sme vytvorili program, ktorý používa kryptografické funkcie z implementácie DSVPN. XOODOO permutáciu sme inicializovali podľa vzoru v DSVPN. Následne sme vygenerovali náhodné dáta a tie sme zašifrovali a dešifrovali. Získané dáta sú znázornené v~tabuľke~\ref{tabmerlokal}. 

\begin{table}[h!]
	\centering
	\resizebox{\textwidth}{!}{%
		\begin{tabular}{c|c|c|c|c}
			\multirow{2}{*}{\bfseries Veľkosť dát [B]} &
			\multicolumn{2}{|c}{\bfseries uc\_encrypt()} &
			\multicolumn{2}{|c}{\bfseries uc\_decrypt()} 
			\\
			& Počet cyklov
			& Čas vykonávania [ms]
			& Počet cyklov
			& Čas vykonávania [ms] 
			\\\hline\hline
			52		 
			& 15 428
			& 0,005 600
			& 16 124
			& 0,005 700
			\\		
			250
			& 50 634
			& 0,017 600
			& 50 866
			& 0,017 700
			\\	
			500		
			& 97 817
			& 0,034 000
			& 97 469
			& 0,033 800
			\\
			1 000		
			& 189 312
			& 0,065 600
			& 189 167
			& 0,065 500
			\\ 
			1 500		
			& 280 691
			& 0,097 100
			& 280 749
			& 0,097 100
			\\
			3 000		
			& 557 322
			& 0,192 700
			& 557 960
			& 0,192 900
			\\
			4 500		
			& 834 997
			& 0,288 600
			& 835 374
			& 0,288 700
			\\
			6 000		
			& 1 110 120
			& 0,383 600
			& 1 109 279
			& 0,383 300
			\\
			7 500		
			& 1 397 974
			& 0,483 100
			& 1 386 258
			& 0,479 000
			\\
			9 000		
			& 1 769 841
			& 0,611 600
			& 1 661 352
			& 0,574 100			
		\end{tabular}%
	}
	\caption{Výsledky z experimentálnych meraní funkcií na šifrovanie a dešifrovanie v prostredí lokálneho zariadenia}
	\label{tabmerlokal}
\end{table} 
Z týchto výsledkov už je jednoznačne, že čas vykonávania šifrovania a dešifrovania, je priamo úmerný veľkosti dát vstupujúcich do XOODOO permutácie. To isté platí aj pre počet vykonaných cyklov. Rádovo sa čas vykonávania počas merania pohyboval pod úrovňou 1 milisekundy a to aj pri šifrovaní dát o veľkosti 9000 bajtov, ktorá predstavuje maximálnu možnú prenosovú veľkosť paketu (MTU). To znamená, že v implementácií DSVPN do (de)šifrovania nevstupujú väčšie dáta. Použitie permutácie XOODOO na zariadení je teda extrémne rýchle a jej použitie pri sieťovom prenose je z používateľského hľadiska zanedbateľné. 

Vytvorený zdrojový kód na meranie šifrovania a dešifrovania je obsahom Prílohy A.4. Konkrétne v priečinku \lstinline|MeasurementsInLocalEnvironment|.
    
\section{Výsledky dosiahnuté pri tvorbe kompatibility \\DSVPN pre OS Windows}
Jedným z cieľov našej práce bolo aj rozšírenie pôvodného riešenia DSVPN o~kompatibilitu pre OS Windows. Samozrejmosťou bolo zachovanie pôvodnej kompatibility v rámci ostatných OS. V princípe sa nám podarilo vytvoriť tunelové spojenie medzi VPN klientom a VPN serverom so zabezpečeným prenosom dát. Taktiež je možné nadviazať spojenie nie len medzi zariadeniami s OS Windows, ale vieme spojiť ľubovoľné OS z množiny podporovaných DSVPN.  

Problém riešenia nastáva pri smerovaní paketov naspäť k pôvodnému vlastníkovi. Pomocou sieťových príkazov a monitoringu premávky sa nám nepodarilo doteraz zistiť, prečo sa odpoveď na pôvodnú požiadavku nedokáže vrátiť na~zariadenie. Tento úkon sa deje aj napriek tomu, že pri monitoringu vidíme ako paket s odpoveďou opúšťa zariadenie a prichádza na to, ktoré má dostať odpoveď. Súčasná hypotéza je, že nám uniká niečo pri smerovaní paketu z pozície OS Windows. Tento problém sa budeme snažiť ešte v blízkej budúcnosti vyriešiť, nakoľko by sa nemalo jednať o veľký problém. Čitateľ bude mať prístup ku aktuálnemu kódu pomocou stránky github.   