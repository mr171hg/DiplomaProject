% !TEX root = ../thesis.tex

\chaptermark{Úvod}
\phantomsection
\addcontentsline{toc}{chapter}{Úvod}

\chapter*{Úvod}
Virtuálna privátna sieť, z ang. \textit{Virtual Private Network} (ďalej \acrshort{vpn}), sa stala bežnou a veľmi využívanou technológiou na zabezpečenie sieťovej komunikácie. S VPN sa stretávame takmer v každej sfére. Domácnosti ju zvyknú používať na získanie prístupu k im nedostupným zdrojom. Vo sfére biznisu zasa s cieľom najlepšieho zabezpečenia dát v pomere s rýchlosťou, ktorú VPN implementácia poskytuje. 

Cieľom práce je priblížiť čitateľovi základné informácie o VPN sieti pomocou postupnej charakteristiky, klasifikácie na základe viacerých aspektov a opisu už existujúcich VPN protokolov. Následne špecifikujeme dôležitý prvok kvalitnej VPN siete, zabezpečenie pomocou kryptografie. Konkrétne sa zameriame na pomerne novú podkategóriu, tzv. ľahkú kryptografiu, z ang. \textit{Lightweight Cryptography}. Tento pojem v práci charakterizuje. Z ľahkej kryptografie sme si za účelom opisu vybrali kryptografickú perrmutáciu XOODOO \cite{xcb}. XOODOO tvorí základ kryptografického balíka Xoodyak \cite{tkecak}, jedného z finalistov štandardizačného procesu Národného inštitútu pre štandardy a technológie (NIST) v kategórií ľahkej kryptografie. V práci ju opíšeme a vysvetlíme možnosti jej použitia. Následne použijeme permutáciu za účelom zabezpečenia jednoduchej siete. VPN sieť vytvoríme pomocou voľne dostupného programu DSVPN napísaného v jazyku C. Program realizuje autentizované šifrovanie s využitím XOODOO permutácie. Experimentálne overíme funkčnosť a implementáciu VPN v DSVPN zanalyzujeme. Následne doplníme zdrojový kód o kompatibilitu s operačným systémom Windows. Vykonané zmeny opíšeme. Súčasťou práce je aj experimentálne meranie počtu potrebných cyklov a rýchlosti šifrovacieho algoritmu počas bežnej prevádzky. 

Prvá kapitola sa venuje charakteristike VPN sietí. Druhá kapitola obsahuje informácie o ľahkej kryptografií so zameraním sa na kryptografickú permutáciu XOODOO. Tretia kapitola informuje čitateľa o použitých nástrojoch, prostredí a~konfigurácií virtuálnych strojov (z ang. \textit{Virtual Machines}, ďalej \acrshort{vm}). Štvrtá kapitola obsahuje analýzu DSVPN a informácie o rozšírení jej kompatibility do operačného systému (z ang. \textit{Operating System}, ďalej OS) Microsoft Windows. V piatej kapitole zhrnieme dosiahnuté výsledky z meraní a experimentov. Spomenuté činnosti budú vykonané v prostredí virtuálnych strojov na dvojici obrazov operačných systémov Microsoft Windows a Linux s distribúciou Ubuntu. Pri tvorbe, úprave a preklade zdrojového kódu použijeme jazyk C s GCC prekladačom. 



