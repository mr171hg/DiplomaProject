% !TEX root = ../thesis.tex

\chaptermark{Úvod}
\phantomsection
\addcontentsline{toc}{chapter}{Úvod}

\chapter*{Úvod}
Virtuálna privátna sieť, tiež známa ako \acrshort{vpn}, sa stala bežnou a veľmi využívanou technológiou na zabezpečenie sieťovej premávky. S VPN sa stretávame takmer v každej sfére. Domácnosti ju zvyknú používať na získanie prístupu k im nedostupným zdrojom. Vo sfére biznisu zasa s cieľom najlepšieho zabezpečenia dát v pomere s rýchlosťou, ktorú VPN implementácia poskytuje. 

Cieľom práce je priblížiť čitateľovi informácie o VPN sieti. Postupnou charakteristikou, klasifikáciou na základe viacerých aspektov a opisom už existujúcich VPN protokolov. Následne špecifikujeme dôležitý prvok kvalitnej VPN siete, zabezpečenie pomocou kryptografie. Konkrétne sa zameriame na pomerne novú podkategóriu ľahká kryptografia. Tento pojem v práci charakterizuje. Z ľahkej kryptografie sme si za účelom opisu vybrali kryptografickú perrmutáciu XOODOO. XOODOO tvorí základ kryptografického balíka Xoodyak, jedného z finalistov štandardizačného procesu Národného inštitútu pre štandardy a technológie (NIST) v kategórií ľahkej kryptografie. V práci ju opíšeme a vysvetlíme možnosti jej použitia. Následne sa presunieme k jej praktickému použitiu za účelom zabezpečenia jednoduchej siete. VPN sieť vytvoríme pomocou voľne dostupného programu DSVPN napísaného v jazyku C. Program realizuje autentizované šifrovanie s využitím XOODOO permutácie. Experimentálne overíme funkčnosť. Krok po kroku implementáciu VPN v DSVPN zanalyzujeme. Následne doplníme zdrojový kód o kompatibilitu s operačným systémom Windows. Vykonané zmeny opíšeme. Súčasťou práce je aj experimentálne meranie počtu potrebných cyklov a rýchlosti šifrovacieho algoritmu počas bežnej prevádzky. 

Dosiahnuté výsledky práce vyhodnotíme v samotnej kapitole tejto práce. Spomenuté činnosti budú vykonané v prostredí virtuálnych strojov na dvojici obrazov operačných systémov Microsoft Windows a Linux s distribúciou Ubuntu. Pri tvorbe, úprave a preklade zdrojového kódu použijeme jazyk C s GCC prekladačom. 



