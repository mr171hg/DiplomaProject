%% -----------------------------------------------------------------
%% This file uses UTF-8 encoding
%%
%% For compilation use following command:
%% latexmk -pdf -pvc -bibtex thesis
%%
%% -----------------------------------------------------------------
%%                                     _     _      
%%      _ __  _ __ ___  __ _ _ __ ___ | |__ | | ___ 
%%     | '_ \| '__/ _ \/ _` | '_ ` _ \| '_ \| |/ _ \
%%     | |_) | | |  __/ (_| | | | | | | |_) | |  __/
%%     | .__/|_|  \___|\__,_|_| |_| |_|_.__/|_|\___|
%%     |_|                                          
%%
%% -----------------------------------------------------------------
\documentclass{kithesis}
\thesisspec{figures/zadavaci-list.png}
% Additional packages
\usepackage[main=slovak,english]{babel}
\usepackage{listings} 
\usepackage{amsmath} 
\usepackage{multirow} 
\usepackage{booktabs}
\usepackage{indentfirst}
\usepackage{dirtree}
\usepackage{hyperref}
\usepackage{amssymb}
  % for source code
% Listings settings
% See for details: https://en.wikibooks.org/wiki/LaTeX/Source_Code_Listings
\lstset{              % nastavení
	%	Definice jazyka použitého ve výpisech
	%    language=[LaTeX]{TeX},	% LaTeX
	%	language={Matlab},		% Matlab
	language={C},           % jazyk C
	basicstyle=\ttfamily,	% definice základního stylu písma
	tabsize=2,			% definice velikosti tabulátoru
	mathescape=<true>,
	inputencoding=utf8,         % pro soubory uložené v kódování UTF-8
	columns=fixed,  %fixed nebo flexible,
	fontadjust=true %licovani sloupcu
	extendedchars=true,
	literate=%  definice symbolů s diakritikou
	{á}{{\'a}}1
	{č}{{\v{c}}}1
	{ď}{{\v{d}}}1
	{é}{{\'e}}1
	{ě}{{\v{e}}}1
	{í}{{\'i}}1
	{ň}{{\v{n}}}1
	{ó}{{\'o}}1
	{ř}{{\v{r}}}1
	{š}{{\v{s}}}1
	{ť}{{\v{t}}}1
	{ú}{{\'u}}1
	{ů}{{\r{u}}}1
	{ý}{{\'y}}1
	{ž}{{\v{z}}}1
	{Á}{{\'A}}1
	{Č}{{\v{C}}}1
	{Ď}{{\v{D}}}1
	{É}{{\'E}}1
	{Ě}{{\v{E}}}1
	{Í}{{\'I}}1
	{Ň}{{\v{N}}}1
	{Ó}{{\'O}}1
	{Ř}{{\v{R}}}1
	{Š}{{\v{S}}}1
	{Ť}{{\v{T}}}1
	{Ú}{{\'U}}1
	{Ů}{{\r{U}}}1
	{Ý}{{\'Y}}1
	{Ž}{{\v{Z}}}1
}
\def\lstlistingname{Zdrojový kód}
\lstset{mathescape=<true>}
% Variables 
%zadanie z maisu
%\thesisspec{figures/meno.png} 

\title{Lightweight cryptography in VPN networks \br)}{Ľahká kryptografia vo VPN sieťach\br }


\author{Bc. Marek}{Rohač}
\supervisor{prof. Ing. Miloš Drutarovský, CSc.} %veduci prace
\consultant{} %konzultant
\college{Technical University of Košice}{Technická univerzita v Košiciach} %univerzita
\faculty{Faculty of Electrical Engineering and informatics}{Fakulta elektrotechniky a informatiky} %fakulta
\department{Department of electronies and multimedia telecommunications}{Katedra elektroniky a multimediálnych telekomunikácií} %katedra
\departmentacr{DEMT}{KEMT} % skratka katedry
\thesis{Master thesis}{Diplomová práca} %typ prace
\submissiondate{ 21}{ 4}{ 2023}
\fieldofstudy{Informatika}
\studyprogramme{Počítačové siete}
%\city{Košice}
\keywords{DSVPN, Lightweight Cryptography, Linux, VPN, Windows, XOODOO}{DSVPN, Ľahká kryptografia, Linux, VPN, Windows, XOODOO}
\declaration{Vyhlasujem, že som záverečnú prácu vypracoval samostatne s použitím uvedenej odbornej literatúry.}

\abstract{The aim of the work is to familiarize the reader with the issue of VPN networks. In the work, we described the basic principles of VPN operation, classification based on several aspects and characteristics of several known VPN protocols. Subsequently, we focused on the characteristics of the concept of lightweight cryptography. We have described the XOODOO cryptographic permutation and the possibilities of its use. In the practical part, we analyzed, demonstrated and experimentally measured the implementation of XOODOO permutation in a simple VPN network. The VPN network is created by the freely available program DSVPN. It is written in C and contains open source code. As part of the practical part, we also focused on expanding compatibility in the DSVPN program for Windows OS. At the end of the work, we summarized the achieved results and offer the reader possibilities for further expansion of the work.
}{Cieľom práce je oboznámiť čitateľa s problematikou VPN sieti. V práci sme opísali základné princípy fungovania VPN, klasifikáciu na základe viacerých aspektov a charakteristiku viacerých známych VPN protokolov. Následne sme sa zamerali na charakteristiku pojmu ľahká kryptografia. Opísali sme kryptografickú permutáciu XOODOO a možnosti jej použitia. V praktickej časti sme analyzovali, demonštrovali a experimentálne odmerali implementáciu XOODOO permutáciu v jednoduchej VPN sieti. VPN sieť vytvára voľne dostupný program DSVPN. Je napísaný v jazyku C a obsahuje otvorený zdrojový kód. V rámci praktickej časti sme sa zamerali aj na rozšírenie kompatibility v programe DSVPN pre OS Windows. V závere práce sme zhrnuli dosiahnuté výsledky a ponúkame čitateľovi možnosti ďalšieho rozšírenia práce.}


\acknowledgment{Na~tomto mieste by~som rád poďakoval svojmu vedúcemu práce \textit{prof. Ing. Milošovi Drutarovskému, CSc.} za jeho čas a odborné vedenie počas celého riešenia záverečnej práce.}

% if you want to work only on selected chapters
%\includeonly{chapters/analyza} %,chapters/synteza}
% Load acronyms
% Acronyms
% ========
%
% An acronym is a word formed from the initial letters in a phrase. 
%
% Acronym Definition Exapmle:
% ---------------------------
% \newacronym{gcd}{GCD}{Greatest Common Divisor}
% \newacronym{dry}{DRY}{Don't Repeat Yourself}
%
% Usage:
% ------
% You can use these three options:
% 
% \acrlong{}  
%   Displays the phrase which the acronyms stands for. Put the label of the acronym inside the braces. In the example, \acrlong{gcd} prints Greatest Common Divisor. 
%
% \acrshort{} 
%   Prints the acronym whose label is passed as parameter. For instance, \acrshort{gcd} renders as GCD. 
%
% \acrfull{ } 
%   Prints both, the acronym and its definition. In the example the output of \acrfull{dry} is Don't Repeat Yourself (DRY). 
% 
% For more information see:
% -------------------------
% * https://www.sharelatex.com/learn/Glossaries 
% * https://en.wikibooks.org/wiki/LaTeX/Glossary
%


\newacronym{vpn}{VPN}{Virtual Private Network}
\newacronym{vm}{VM}{Virtual Machine}
\newacronym{vb}{VB}{Virtual Box}
\newacronym{lts}{LTS}{Long Term Support}
\newacronym{os}{OS}{Operating System}
\newacronym{tcp}{TCP}{Transmission Control Protocol}
\newacronym{oss}{OSS}{Operating System Server}
\newacronym{osc}{OSC}{Operating System Client}
\newacronym{ip}{IP}{Internet Protocol}
\newacronym{ftp}{FTP}{File Transfer Protocol}
\newacronym{tls}{TLS}{Transport Layer Security}
\newacronym{ssl}{SSL}{Secure Socket Layers}
\newacronym{gw}{GW}{GateWay}
\newacronym{fw}{FW}{FireWall}
\newacronym{pdv}{PDV}{Packet Delay Variation}
\newacronym{aes}{AES}{Advanced Encryption Standard}
\newacronym{fskd}{FSKD}{Full-State Keyed Duplex}
\newacronym{tcpip}{TCP/IP}{Transmission Control Protocol/Internet Protocol reference model}
\newacronym{osi}{OSI}{Open Systems Interconnection reference model}
\newacronym{http}{HTTP}{HyperText Transfer Protocol}
\newacronym{smtp}{SMTP}{Simple Mail Transfer Protocol}
\newacronym{udp}{UDP}{User Datagram Protocol}
\newacronym{l}{L}{Layer}
\newacronym{pptp}{PPTP}{Point-to-Point Tunneling Protocol}
\newacronym{p2p}{PPP}{Point-to-Point Protocol}
\newacronym{nas}{NAS}{Network Access Server}
\newacronym{gre}{GRE}{Generic Routing Encapsulation}
\newacronym{chap}{CHAP}{Challenge Handshake Authentication Protocol}
\newacronym{mschap}{MS-CHAP}{MicroSoft Challenge Handshake Authentication Protocol}
\newacronym{pap}{PAP}{Password Authentication Protocol}
\newacronym{l2tp}{L2TP}{Layer 2 Tunneling Protocol}
\newacronym{mpls}{MPLS}{Multiprotocol Label Switching}
\newacronym{rfc}{RFC}{Request For Comments}
\newacronym{l2f}{L2F}{Layer 2 Forwarding protocol}
\newacronym{lns}{LNS}{L2TP Network Server}
\newacronym{lac}{LAC}{L2TP Access Concentrator}
\newacronym{poe}{PPPoE}{PPP over Ethernet}
\newacronym{nat}{NAT}{Network Address Translation}
\newacronym{ipsec}{IPSec}{Internet Protocol Security}
\newacronym{isakmp}{ISAKMP}{Internet Security Association and Key Management Protocol}
\newacronym{sa}{SA}{Security Associations}
\newacronym{ah}{AH}{Authentication Header}
\newacronym{esp}{ESP}{Encapsulating Security Payloads}
\newacronym{dhke}{DHKE}{Diffie–Hellman Key Exchange}
\newacronym{ecdh}{ECDH}{Elliptic-Curve Diffie–Hellman}
\newacronym{rsa}{RSA}{Rivest–Shamir–Adleman}
\newacronym{ecdsa}{ECDSA}{Elliptic Curve Digital Signature Algorithm}
\newacronym{https}{HTTPS}{HyperText Transfer Protocol Secure}
\newacronym{sstp}{SSTP}{Secure Socket Tunneling Protocol}




%% -----------------------------------------------------------------
%%          _                                       _   
%%       __| | ___   ___ _   _ _ __ ___   ___ _ __ | |_ 
%%      / _` |/ _ \ / __| | | | '_ ` _ \ / _ \ '_ \| __|
%%     | (_| | (_) | (__| |_| | | | | | |  __/ | | | |_ 
%%      \__,_|\___/ \___|\__,_|_| |_| |_|\___|_| |_|\__|
%%                                                      
%% -----------------------------------------------------------------

\begin{document}
%% Title page, abstract, declaration etc.:
\frontmatter{}

%% List of code listings, if you are using package minted
%\renewcommand\lstlistlistingname{Zoznam zdrojových kódov}
%\lstlistoflistings{}

\pagenumbering{arabic}

%% Chapters
% !TEX root = ../thesis.tex

\chaptermark{Úvod}
\phantomsection
\addcontentsline{toc}{chapter}{Úvod}

\chapter*{Úvod}
Virtuálna privátna sieť, tiež známa ako \acrshort{vpn}, sa stala bežnou a veľmi využívanou technológiou na zabezpečenie sieťovej premávky. S VPN sa stretávame takmer v každej sfére. Domácnosti ju zvyknú používať na získanie prístupu k im nedostupným zdrojom. Vo sfére biznisu zasa s cieľom najlepšieho zabezpečenia dát v pomere s rýchlosťou, ktorú VPN implementácia poskytuje. 

Cieľom práce je priblížiť čitateľovi informácie o VPN sieti. Postupnou charakteristikou, klasifikáciou na základe viacerých aspektov a opisom už existujúcich VPN protokolov. Následne špecifikujeme dôležitý prvok kvalitnej VPN siete, zabezpečenie pomocou kryptografie. Konkrétne sa zameriame na pomerne novú podkategóriu ľahká kryptografia. Tento pojem v práci charakterizuje. Z ľahkej kryptografie sme si za účelom opisu vybrali kryptografickú perrmutáciu XOODOO. XOODOO tvorí základ kryptografického balíka Xoodyak, jedného z finalistov štandardizačného procesu Národného inštitútu pre štandardy a technológie (NIST) v kategórií ľahkej kryptografie. V práci ju opíšeme a vysvetlíme možnosti jej použitia. Následne sa presunieme k jej praktickému použitiu za účelom zabezpečenia jednoduchej siete. VPN sieť vytvoríme pomocou voľne dostupného programu DSVPN napísaného v jazyku C. Program realizuje autentizované šifrovanie s využitím XOODOO permutácie. Experimentálne overíme funkčnosť. Krok po kroku implementáciu VPN v DSVPN zanalyzujeme. Následne doplníme zdrojový kód o kompatibilitu s operačným systémom Windows. Vykonané zmeny opíšeme. Súčasťou práce je aj experimentálne meranie počtu potrebných cyklov a rýchlosti šifrovacieho algoritmu počas bežnej prevádzky. 

Dosiahnuté výsledky práce vyhodnotíme v samotnej kapitole tejto práce. Spomenuté činnosti budú vykonané v prostredí virtuálnych strojov na dvojici obrazov operačných systémov Microsoft Windows a Linux s distribúciou Ubuntu. Pri tvorbe, úprave a preklade zdrojového kódu použijeme jazyk C s GCC prekladačom. 




% !TEX root = ../thesis.tex

\chapter{Virtual Private Network -- VPN}\label{1}
Virtuálna privátna sieť (ďalej \textbf{\acrshort{vpn}}) je jeden zo spôsobov prepojenia zariadení, tak že internetová komunikácia medzi nimi je privátna, resp. zabezpečená aj v prípade používania nezabezpečenej, verejnej siete. Bezpečnosť spojenia je docielená pomocou kryptografických protokolov v tuneli, ktorý VPN vytvára. Pod pojmom tunel sa v skutočnosti myslí virtuálna zašifrovaná linka, ktorou je dátový paket prenášaný po sieti medzi koncovými zariadeniami. V skutočnosti tunel vzniká pomocou procesu zapuzdrenia dát. V závislosti od toho na akej úrovni OSI referenčného modelu sa pohybujeme. 
Táto technológia patrí aktuálne k najpoužívanejším spôsobom pripojenia sa medzi 2 rôznymi internetovými doménami. Najčastejší výskyt je možné sledovať v korporačnom prostredí, pričom cieľom je rozšírenie možností bezpečného pripojenia sa k firemnej sieti. Vzhľadom na firemné tajomstvá je nutné aby bolo takéto spojenie bezpečné a zamestnanci sa mohli pripojiť z rôznych miest. Vďaka uvedeným vlastnostiam je následne možná aj práca z~domu (z~ang. \textit{Home office}), ktorá môže byť benefitom pre obe strany. Ukážka použitia VPN je znázornená pomocou \ref{vpndemo} a \ref{vpnfancy}. 

\begin{figure}[!h]
	\centering
	\includegraphics[width=.8\textwidth]{figures/VPNdemo}
	\caption{Ukážka typického VPN pripojenia}
	\label{vpndemo}
\end{figure}
% TODO: \usepackage{graphicx} required
\begin{figure}
	\centering
	\includegraphics[width=0.7\linewidth]{figures/vpn_fancy}
	\caption{vpn fancy obrazok}
	\label{vpnfancy}
\end{figure}

https://www.webhostingsecretrevealed.net/blog/security/how-vpn-works/

\section{Výhody a nevýhody VPN sieti}
neviem ci to bude rozsahovo treba na samostatnu podkapitolu
mozno len doplnit informacie do textu zaciatku.

V prípade nestabilného spojenia, je možný vysoký výskyt straty paketov. Dôvodom je použitie TCP protokolu. Takýto problém je vhodné riešiť, tak sa odporúča prejsť na protokol 

\section{Charakteristika a definovanie pojmov}
Obsahom tejto podkapitoly je zavedenie a následne stručná charakteristika pojmov, potrebných na pochopenie problematiky VPN sieti.
\subsection{Tunel}
https://en.wikipedia.org/wiki/Tunneling\_protocol
https://en.wikipedia.org/wiki/TUN/TAP

wintun --> build https://learn.microsoft.com/en-us/windows-hardware/drivers/download-the-wdk
\subsection{GRE - Generic Routing Encapsulation}\label{gre}
??
https://www.rfc-editor.org/info/rfc1701
https://www.ietf.org/rfc/rfc1702.txt

\subsection{IKEv2 Internet Key Exchange version 2}
https://en.wikipedia.org/wiki/Internet\_Key\_Exchange

\subsection{Transport Layer Security -- TLS}
TCP protokol sam o sebe nezabezpečí dáta, ku ktorým sa pridáva TCP hlavička. Dôsledkom toho vznikli viacere protokoly slúžiace na autentizované šifrovanie dát. Najznámejší je protokol zabezpečenia prenosu -- \acrshort{tls} a IPSec.

Zabezpečenie dát bolo prvotne vykonávané pomocou protokolu \textit{Secure Sockets Layer} -- \acrshort{ssl}. Tento spôsob používa certifikáty na odšifrovanie dát. SSL malo od svojho vytvorenia dlhý vývoj, ktorý smeroval až k doteraz najpoužívanejšiemu TLS vo verzii 1.3. Inými slovami, TLS protokol je nástupca SSL pričom obsahuje rôzne úpravy a vylepšenia najmä z hľadiska rýchlosti. Zároveň sa v dnešnej dobe neodporúča používanie SSL protokolu. Dôsledkom optimalizácií je, že klienta komunikujúci so serverom cez HTTPS protokol s TLS 1.3 je rýchlejší ako v prípade použitia nešifrovaného HTTP variantu. 

TLS pracuje niekde na pomedzi aplikačnej a transportnej vrstvy. Spôsob spracovania dát je zobrazený pomocou schémy \ref{ssl} s SSL operáciami, prebratej z \cite{biks}. 
\begin{figure}[!ht]
	\centering
	\includegraphics[width=0.7\linewidth]{figures/ssl}
	\caption{Prehľad operácií v SSL protokole}
	\label{ssl}
\end{figure}

Viac informácií o TLS protokole, jednotlivých verziách a optimalizáciách je možné nájsť na \cite{tls}. 

viac viac viac

\subsection{\acrfull{p2p}}
https://en.wikipedia.org/wiki/Point-to-Point\_Protocol

\subsection{\acrfull{mpls}}
\acrshort{mpls} by sme do slovenčiny mohli preložiť ako Multi-protokolové prepínanie štítkov. Samotný protokol pridáva k paketom tzv. MPLS hlavičku. Jej obsahom sú štítky\footnote{z ang. \textit{labels}}, na základe ktorých dochádza k smerovaniu a preposielaniu paketov. Týmto spôsobom nie je nutná ďalšia analýza paketov. Štítky neidentifikujú koncové body, ale virtuálne spojenia medzi viacerými uzlami. Označenie multi-protokolový má, pretože dokáže zapuzdrovať viacero sieťových protokolov. MPLS pracuje na pomedzi L2 a L3 vrstvy. Často sa zvykne označovať ako L2.5 protokol. Viac o tomto protokole je možné si prečítať na \cite{mpls}, \cite{mplstuke} a \cite{mplsrfc}.

\subsection{Charakteristika referenčných modelov}\label{crm}
V rámci  podkapitoly \ref{rm} je potrebné pred samotnou klasifikáciou vysvetliť pojmy Referenčný model prepojenia otvorených systémov\footnote{z ang. \textit{Open Systems Interconnection reference model}} (ďalej \acrshort{osi}) a Model opisujúci balíky internetových protokolov, známy ako, \acrshort{tcpip} referenčný model.

Úlohou uvedených referenčných modelov je vizualizácia postupu spracovania dát od používateľa až k ich odoslaniu zo zariadenia\footnote{z ang. \textit{end-to-end data communication}}. Pojem spracovanie dát znamená opísanie toho ako dochádza v jednotlivých abstraktných vrstvách k pretransformovaniu používateľských dát na súbor jednotiek a núl, ktoré sú následne odoslané do iného zariadenia. 

\acrshort{osi} Model vznikol v skorších fázach evolúcie počítačových sietí. Vychádza skôr z teoretického než praktického prístupu. Pozostáva zo 7 abstraktných vrstiev\footnote{z ang. \textit{layer}}:

\begin{enumerate}
	\item{\textbf{fyzická vrstva}} -- z ang. \textit{Physical Layer} (ďalej L1), 
	\item{\textbf{spojová vrstva}} -- z ang. \textit{Data Link Layer} (ďalej L2),
	\item{\textbf{sieťová vrstva}} -- z ang. \textit{Network Layer} (ďalej L3),
	\item{\textbf{transportná vrstva}} -- z ang. \textit{Transport Layer} (ďalej L4),
	\item{\textbf{relačná vrstva}} -- z ang. \textit{Session Layer} (ďalej L5),
	\item{\textbf{prezenčná vrstva}} -- z ang. \textit{Presentation Layer} (ďalej L6),
	\item{\textbf{aplikačná vrstva}} -- z ang. \textit{Application Layer} (ďalej L7).
\end{enumerate}

V schéme \ref{osi} sa nachádzajú stručne opísané jednotlivé činnosti vrstiev.
\begin{figure}[!h]
	\centering
	\includegraphics[width=0.8\textwidth]{figures/osi}
	\caption{https://www.comptia.org/blog/open-systems-interconnection-reference-model - dobry zdroj na rozsirenie}
	\label{osi}
\end{figure}


Čím je väčšie číslo vrstvy tým bližšie sa dáta nachádzajú pri používateľovi. Vďaka uvedeným vlastnostiam je tento model vhodnejší pri začiatku štúdia spracovania sieťových dát v počítači. Z rovnakého dôvodu sa taktiež viac stretávame s jeho použitím pri opise funkcionality riešenia ako s  \acrshort{tcpip} novším modelom.  Viac informácií o problematike nájde čitateľ v \cite{osi}.

Na druhej strane \acrshort{tcpip} model vznikol z praktického prístupu. Hlavný rozdiel je v počte abstraktných vrstiev, ktorý je v prípade \acrshort{tcpip} zmenšený na 4 vrstvy \cite{tcpip}. 
TCP/IP model je zobrazený pomocou schémy \ref{tcpipprot}. V uvedenej schéme sú znázornené aj niektoré z protokolov, ktoré sa na jednotlivých vrstvách používajú.

\begin{figure}[!ht]
	\centering
	\includegraphics[width=0.9\textwidth]{figures/tcpipprot}
	\caption{Schéma TCP/IP modelu s niektorými protokolmi}
	\label{tcpipprot}
\end{figure}

\textbf{Aplikačná vrstva (L5-7)} zahŕňa protokoly používané väčšinou aplikácií na poskytovanie užívateľských služieb alebo výmenu aplikačných dát cez sieťové pripojenia, ktoré je vytvorené protokolmi na nižšej úrovni.  Spája vrstvy L5 až L7 OSI modelu. Príklady známych protokolov aplikačnej vrstvy sú \acrfull{http}, \acrfull{ftp}, \acrfull{smtp} a iné. Údaje, resp. dáta sú pri spracovaní kódované podľa L4 protokolov. Sú zapuzdrené do protokolových jednotiek transportnej vrstvy, tzv. \textbf{segmentov}.  

\textbf{Transportná vrstva (L4)} vytvára základné dátové kanály, ktoré aplikácie používajú na výmenu dát. Vrstva vytvára konektivitu medzi hostiteľmi, ktorá je nezávislá od siete, štruktúry užívateľských dát a smerovacích informácií. Konektivita na transportnej vrstve môže byť kategorizovaná ako orientovaná na spojenie, implementovaná v protokole \acrshort{tcp}, alebo \acrshort{udp} bez orientácie na spojenie. Uvedené protokoly sú stručne charakterizované nižšie, v tejto podkapitole. 

Protokoly v tejto vrstve zabezpečujú:
\begin{itemize}
	\item{riadenie chýb} -- z ang. \textit{error control} \cite{ec},
	\item{segmentáciu dát} -- z ang. \textit{segmentation} \cite{sd},
	\item{riadenie toku dát} -- z ang. \textit{flow control} \cite{fc},
	\item{riadenie preťaženia} -- z ang. \textit{congestion control} \cite{cc},
	\item{adresovanie aplikácií} -- z ang. \textit{application addressing} \cite{aa}.
\end{itemize}
Výstupom transportnej vrstvy sú segmenty, ktoré sú spracované v ďalšej vrstve referenčného modelu. 

\textbf{Internetová, resp. Sieťová vrstva (L3)} je zodpovedná za odosielanie paketov cez jednu alebo viac sietí. S touto funkcionalitou internetová vrstva umožňuje sieťovanie, prepojenie rôznych IP sietí a v podstate vytvára internet. Z L4 segmentov tvorí pakety, tak že pridá informácie potrebné na ďalšie smerovanie. 

\textbf{Linková vrstva (L2)} sa používa na presun paketov medzi rozhraniami internetovej vrstvy dvoch rôznych hostiteľov na rovnakom linku. Procesy vysielania a prijímania paketov na linke je možné konfigurovať. Zariadenia nazývané prepínače \footnote{z ang. \textit{switch}}, vykonávajú rámcovania\footnote{z ang. \textit{frame}}. Pripravujú pakety z L3 vrstvy na prenos pridaním ďalších informácií. Týmto úkonom vzniknú rámce. Tie sa prenášajú do fyzickej vrstvy a cez prenosové médium až k hostiteľovi. Fyzická vrstva predstavuje prvok, cez ktoré sú dáta prenášane. Napríklad optický a ethernetový kábel. 

Vyššie uvedené procesy prípravy dát sú znázornené pomocou schémy \ref{p1} a \ref{p2}.

\begin{figure}[!h]
	\centering
	\includegraphics[width=0.7\linewidth]{figures/prenos1z2}
	\caption{spoj tento obrazok s nasledujucim do jedneho}
	\label{p1}
\end{figure}
\begin{figure}
	\centering
	\includegraphics[width=0.7\linewidth]{figures/prenos2z2}
	\caption{spoj tento obrazok s predchadzajucim do jedneho, plus \ref{osi}}
	\label{p2}
\end{figure}
\newpage
\textbf{Protokol riadenia prenosu} (z ang. \textit{Transmission Control Protocol}, ďalej \acrshort{tcp}) je komunikačný protokol orientovaný na nadviazanie a udržanie sieťového spojenia medzi zariadeniami. Môže byť použitý pri úlohe príjímateľa aj odosielateľa (z ang. \textit{full-duplex}). Úlohou je spoľahlivý prenos dát medzi komunikantmi. Odoslanie a príjem dát je v rovnakom poradí. Protokol zároveň obsahuje mechanizmy na kontrolu výskytu chýb. Svoj názov má podľa dvoch najdôležitejších protokolov:
\begin{itemize}
	\item{Protokol riadenia prenosu} -- z ang. \textit{\acrlong{tcp}},
	\item{Internet protokol} -- z ang. \textit{\acrlong{ip}}. 
\end{itemize}

Na začiatku 21. storočia je 95\% paketov používaných na internete TCP. Bežné aplikácie používajúce TCP sú webové (HTTP/HTTPS protokoly), slúžiace na e-mailovú komunikáciu (SMTP/POP3/IMAP) a prenos súborov (z ang. \textit{File Transfer Protocol -- \acrshort{ftp}}). Minimálna dĺžka hlavičky TCP je 20 bajtov a maximálna dĺžka 60 bajtov. Po pridaní údajov TCP hlavičky k prenášaným dátam, vzniká tzv. \textit{segment}.

V súčasnosti je možne TCP protokol implementovať softvérovo aj hardvérovo. Pri prvom z uvedených je problémom závislosť na OS a následne aj vysoká vyťaženosť procesora pri príprave a spracovaní dát. Pri hardvérovom riešení je výhodou optimalizácia a implementácia bez potreby dodatočnej úpravy OS. Hardvérové implementácie sa realizujú pomocou koprocesorov umiestnených vo vnútri procesora. Následkom toho môžeme dnes bežne pozorovať umiestnenie spomenutých zariadení na našich zariadeniach.

Podrobnejšie informácie o TCP protokole je možné nájsť na \cite{tcp2}. V uvedenej publikácií sa nachádza opis TCP hlavičky, metódy nadviazania a ukončenia spojenia. Obdobne je spomenuté ako dochádza k prenosu dát pomocou sekvenčných čísel. Ak má používateľ nejasnosti v fungovaní TCP protokolu, odporúča sa danú publikáciu prečítať.

\textbf{Používateľský datagramový protokol} (z ang. \textit{User Datagram Protocol}, ďalej \acrshort{udp}) je jedným zo základných komunikačných \acrshort{ip} protokolov. Používa sa na odosielanie správ iným hostiteľom v sieti. Správy sú prenášané ako datagramy v paketoch. \acrshort{udp} nevyžaduje predchádzajúcu komunikáciu na nastavenie komunikačných kanálov alebo dátových ciest. Používa jednoduchý komunikačný model bez spojenia s minimom protokolových mechanizmov. Poskytuje kontrolné súčty pre integritu údajov a čísla portov na adresovanie rôznych funkcií v zdroji a cieli datagramu. 

Narozdiel od \acrshort{tcp}) neposkytuje žiadnu záruka doručenia správy alebo duplicitnej ochrany. 
\acrshort{udp}) je vhodný na účely, kde kontrola a oprava chýb buď nie sú potrebné, alebo sa vykonávajú v aplikácii. Aplikácie citlivé na čas často používajú UDP, pretože zahadzovanie paketov je vhodnejšie ako čakanie na pakety oneskorené v dôsledku opätovného prenosu. Príklad použitia môžu byť streamovacie služby. 
Podrobnejšie informácie o \acrshort{udp}) protokole nájde čitateľ v \cite{udp}.

\section{Klasifikácia VPN sietí}
V súčasnosti má čitateľ k dispozícií veľa rôznych internetových zdrojov o problematike VPN. Uvedené sú rôzne možnosti klasifikácie VPN siete. V rámci tejto práce klasifikujeme VPN siete podľa logickej topológie, použitých protokolov a vrstiev, na ktorých sú postupy aplikované. Obsahom tejto podkapitoly je rozdelenie a opis jednotlivých typov VPN sieti. V závere kapitoly čitateľ nájde sumárne rozdelenie VPN sieti v tejto práci, znázornené pomocou schémy.  
\subsection{Rozdelenie VPN sieti podľa logickej topológie}
 Podľa topológie, v ktorej VPN spojenie prebieha rozdeľujeme VPN do 3 kategórií: 
\begin{itemize}
	\item \textbf{VPN rovný s rovným} -- \textit{z ang. Peer to Peer VPN},
	\item \textbf{VPN klient a server} -- \textit{z ang. Client to Server VPN}, 
	\item \textbf{VPN sieť so sieťou} -- \textit{z ang. Site to Site VPN}.
\end{itemize}

\subsubsection{VPN rovný s rovným}
Uvedený spôsob vytvára zabezpečený tunel medzi dvoma rovnocennými uzlami, resp. zariadeniami\footnote{z ang. \textit{peers}}, ktorý spoločne komunikujú cez verejnú sieť. Medzi zariadeniami je vytvorený tunel. Každý koniec má priradenú svoju IP adresu. Z uvedeného modelu vyplýva aj následná limitácia. VPN tunel vznikne iba medzi dvoma komunikujúcimi zariadeniami. Z toho dôvodu nie je toto použitie časté. Na obrázku \ref{p2p} je znázornený uvedený typ VPN spojenia. 
\begin{figure}[!ht]
	\centering
	\includegraphics[width=.7\textwidth]{figures/p2p}
	\caption{}
	\label{p2p}
\end{figure}
 
\subsubsection{VPN Klient a Server}
Tento typ spojenia pozostáva z pripojenia medzi nerovnocennými dvoma alebo viacerými zariadeniami. Najjednoduchší model musí pozostávať z jedného VPN servera a VPN klienta. Princíp spočíva vo vytvorení zabezpečeného tunela, ktorý je použitý na prenos dát medzi uvedenými zariadeniami. Zároveň je však možné vytvoriť $1:N$ takýchto spojení. $N$ reprezentuje počet VPN klientov, resp. prepojení, ktorý VPN Server dokáže nadviazať. Tento parameter je závislý najmä od hardvérových prostriedkov daného servera.

Úloha Klienta spočíva v presmerovaní všetkej svojej sieťovej komunikácie cez zabezpečený tunel, ktorý vznikol medzi ním a Serverom. Tento úkon je najčastejšie realizovaný presmerovaním trafiky cez sieťovú bránu\footnote{z ang. \textit{GateWay}} (ďalej \acrshort{gw}). V danom \acrshort{os}, na ktorom VPN klient beží, je teda potrebné zmeniť IP adresu \acrshort{gw} na adresu VPN servera. Vďaka tomu nastane presmerovania komunikácie. Tento úkon je väčšinou realizovaný programovo pomocou aplikácií. Typicky nastolia spojenie medzi Klientom a Serverom. Následne upravia sieťové nastavenia systému. Spomenuté úkony sú vysoko závislé od \acrshort{os} a daného programovacieho jazyka, prostredníctvom ktorého sú úpravy realizované. 
Úloha Servera na druhej strane spočíva vo vytvorení možnosti pripojenia pre jedného alebo viacerých klientov. Následne Server zastupuje klientovu prítomnosť v danej sieti. Teda spracúva požiadavky Klienta a komunikuje s ostatnými zariadeniami. Komunikácia ďalej však už nie je zabezpečená pomocou šifrovania alebo tunelu. Tento fakt je znázornený na obrázku \ref{c2s}.   

\begin{figure}[!ht]
	\centering
	\includegraphics[width=.7\textwidth]{figures/c2s}
	\caption{potrebne prerobenie nech sa to nepodoba na 1.1 alebo vyhodit/upravit 1.1}
	\label{c2s}
\end{figure}
V súčasnosti je tento spôsob považovaný za najviac používaný v oblasti korporátneho sveta. VPN server slúži ako vstupná brána do internej siete. Vďaka tomu je možné sprístupniť zdroje pre používateľov z rôznych oblastí sveta. Používateľ sa taktiež môže stretnúť s pomenovaním model \textbf{uzol k sieti}, resp. z ang \textit{point-to-site}. Obidva pojmy sú správne a predstavujú rovnakú myšlienku zapojenia VPN.  
\subsubsection{Sieť so Sieťou VPN}
VPN model Sieť so Sieťou vytvára zabezpečený tunel medzi dvoma rôznymi sieťami naprieč verejnou sieťou. Model pozostáva z 2 zariadení -- VPN servera a VPN Koncentrátora\footnote{z ang. \textit{VPN Concentrator}}.

VPN Koncentrátor je typ sieťového zariadenia, ktoré poskytuje zabezpečené VPN spojenie a doručenie dát. Zvyčajne je to špecializovaný smerovač\footnote{z ang. \textit{router}}. Dokáže vytvárať veľké množstvo VPN tunelov. Používa sa na nastolenie VPN modelu sieť so sieťou. Funkcionalita koncentrátora pozostáva z: 
\begin{itemize}
	\item{nastolenie a konfiguráciu VPN tunela} -- z ang. \textit{Establish and Configure tunnels},
	\item{autentizáciu používateľa}-- z ang. \textit{Authenticate users},
	\item{priradenie IP adries používateľov k tunelom} -- z ang. \textit{Assign tunnel/IP addresses to users},
	\item{šifrovanie a dešifrovanie dát} -- z ang. \textit{Encrypt and Decrypt data},
	\item{zabezpečiť integritu doručenia} -- z ang. \textit{Ensure end-to-end delivery of data}.
\end{itemize}

Model Sieť so sieťou je používaný najmä pri spojení vedľajšej pobočky s hlavnou, ktoré sa nachádzajú na rozdielnych geografických lokalitách. Pomocou schémy \ref{sts} je znázornený tento model. 

 \begin{figure}[!ht]
 	\centering
 	\includegraphics[width=.7\textwidth]{figures/sts}
 	\caption{rearanzovat}
 	\label{sts}
 \end{figure}  

Informácia z tejto podkapitoly boli čerpané z \cite{vpntech}.  
\subsection{Rozdelenie VPN sieti podľa vrstiev referenčného modelu}\label{rm}
VPN siete môžeme klasifikovať podľa vrstvy (ďalej \acrshort{l}) referenčného modelu, na ktorej dochádza k šifrovaniu dát. Stručná charakteristika referenčných modelov je obsahom podkapitoly \ref{crm}.

Klasifikácia VPN sietí podľa \acrshort{osi} modelu:
\begin{itemize}
	\item{\textbf{L2 VPN}} -- VPN na spojovej vrstve,
	\item{\textbf{L3 VPN}} -- VPN na sieťovej vrstve,
	\item{\textbf{L4 VPN}} -- VPN na transportnej vrstve.
\end{itemize}

Pri implementácií VPN komunikácie medzi zariadeniami je pre pochopenie dôležité určiť aké dáta vstupujú do šifrovacieho algoritmu. Pomocou tejto informácie a klasifikácie vyššie následne dokážeme zaradiť do určitej kategórie. Na schéme \ref{osidata} je možné si všimnúť aké dáta sú výstupom jednotlivých vrstiev. (spoj dva obrazky do jedneho netreba zbytocne dva)

\begin{figure}[!h]
	\centering
	\includegraphics[width=0.7\textwidth]{figures/osidata}
	\caption{zakomponovat k osi vysvetleniu}
	\label{osidata}
\end{figure}

Princíp použitia bloku na šifrovanie je znázornený pomocou schémy \ref{sifr}.

\begin{figure}[!h]
	\centering
	\includegraphics[width=0.7\textwidth]{figures/sifr.jpg}
	\caption{potreba upravy iba na spoemnute vrstvy}
	\label{sifr}
\end{figure}

Zo schémy je viditeľné, že spracovaním nižšej vrstvy dochádza k zväčšeniu bloku dát, ktoré je potrebné šifrovať. To môže negatívne ovplyvniť výslednú rýchlosť celého systému. Okrem uvedenej vlastnosti je dalším problémom prenos vrámci rôznych sieti. Každé zariadenie, ktoré smeruje dáta by muselo repetitívne šifrovať a dešifrovať čím by v danej implementácií vzniklo viac rizikových miest pre potencionálny útok. Z uvedených dôvodov je preto najrozšírenejší spôsob zapojenia šifrovacieho bloku na pomedzi L3 až L4 vrstvy. Smerovanie ostáva bez zásahu a k dešifrovaniu dochádza až na konečnom zariadení, prípadne v aplikácií na samotných dátach.

Typický sa s L2 a L3 VPN sieťami môže používateľ stretnúť na špecializovaných sieťových zariadeniach. Konkrétne na routroch a switchoch. Prvé z uvedených je využívané najmä pri smerovaní, respektíve určení cesty smerom von z lokálnej siete až k cieľovej destinácií. Tento úkon je vykonaný za pomoci aplikácie smerovacích protokolov. Router je možné použiť aj na smerovanie v rámci lokálnej siete. V porovnaní so switchom však nedosahuje porovnateľný výkon. Na druhej strane klaasický switch je možné použiť len vrámci lokálnej siete. V súčasnosti sa používajú aj tzv. L3 switche. V porovnaní s routrom je ich prednosťou vyššia rýchlosť spracovania prichádzajúcich paketov pri väčšom počte pripojených zariadení.        

\section{Protokoly vo VPN sieťach} 
V predchádzajúcej podkapitole sme klasifikovali VPN siete na základe zapojenia kryptografického bloku do OSI referenčného modelu. V tejto podkapitole pomocou uvedeného rozdelenia, zaradíme a charakterizujeme zopár protokolov, ktoré sa typicky používajú vo VPN sieťach. Pred začiatkom by som rád čitateľa upozornil, že aktuálne neexistuje svetový štandard na vytváranie VPN spojení. Dôsledkom toho existuje veľké množstvo rôznych protokolov. Vrámci tejto podkapaitoly si predstavíme niektoré z nich. 

\subsection{\acrfull{pptp}}

?nerozumiem?PPTP enables on-demand, virtual private networks over the Internet or other public TCP/IP-based data networks ??\\

\acrshort{tcpip} sieťový protokol \acrshort{pptp} poskytuje zabezpečuje bezpečný prenos dát z klienta do privátneho servera vo VPN sieti. Jedná sa o starší Microsoft L2 protokol, ktorý bol definovaný v roku 1996. Je rozšírením L2 \acrshort{p2p} protokolu. \acrshort{pptp} zapuzdruje \acrshort{p2p} pakety do  IP datagramov. Následne ich prenáša cez sieť. Typicky pozostáva spojenie z 3 zariadení, ktoré sú znázornené na schéme \ref{ptptun} a \ref{pptpcon}. 

% TODO: \usepackage{graphicx} required
\begin{figure}[h]
	\centering
	\includegraphics[width=0.7\textwidth]{figures/ptptun}
	\caption{Architektura PPTP siete - PPTP je az od nas do PS}
	\label{ptptun}
\end{figure}
% TODO: \usepackage{graphicx} required
\begin{figure}[h]
	\centering
	\includegraphics[width=0.7\textwidth]{figures/pptpcon}
	\caption{lepsi obrazok}
	\label{pptpcon}
\end{figure}

Zabezpečenie prenosu pomocou \acrshort{pptp} pozostáva typicky z 3 po sebe nasledujúcich procesov:
\begin{enumerate}
	\item{PPP pripojenie a komunikácia } -- z ang. \textit{PPP Connection and Communication},
	\item{riadenie spojenia pomocou PPTP protokolu} -- z ang. \textit{PPTP Control Connection},
	\item{prenos dát PPTP tunelom} -- z ang. \textit{PPTP Data Tunneling}.
\end{enumerate}
Prechod medzi procesmi je možný len ak došlo k úspešnému dokončeniu predchádzajúceho kroku. V prvom procese sa PPTP klient pripája k serveru s prístupom na internet\footnote{z ang. \textit{Network Access Server}} (ďalej \acrshort{nas}). Používa sa pri tom PPP protokol, ktorý nastolí spojenie a zašifruje pakety. V druhom kroku následne vznikne TCP pripojenie medzi NAT a PPTP serverom. Použitý je port 1723. Uvedeným postupom nám vznikne medzi zariadeniami PPTP Tunel. Po úspešnom vytvorení konektivity dochádza k zapuzdrovaniu prichádzajúcich zašifrovaných PPP dát do PPTP protokolu a ich prenosu cez tunel. PPTP zapuzdruje dáta do tzv. IP datagramov, ktoré už obsahujú zašifrovaný PPP paket. Datagramy su vytvorené pomocou \acrshort{gre} protokolu, ktorý bol opísaný v \ref{gre}. Na schéme \ref{pptpdat} je znázornená schém PPTP Paketu. 

\begin{figure}
	\centering
	\includegraphics[width=0.7\textwidth]{figures/pptpdat}
	\caption{sede sifrovane biele nie }
	\label{pptpdat}
\end{figure}

PPTP Server po prijatí dáta rozbalí, dešifruje PPP paket a následne ho smeruje v rámci lokálnej siete.

Je dôležité poznamenať, že PPTP klient môže mať priamy prístup na internet. V uvedenom prípade sa nevytvára prvotné PPP spojenie až k internetovému poskytovateľovi.
Viac informácií o protokole PPTP nájde čitateľ na \cite{rfc2637} a \cite{pptp}, ktoré boli zdrojom pri tvorbe tejto podkapitoly.
\subsubsection{Bezpečnosť a použitie protokolu} 
Protokol vznikol v júny 1996. Implementácia poskytuje používateľovi:
\begin{itemize}
	\item{\textbf{Autentizáciu}} -- z ang. \textit{Authentication}, overenie totožnosti používateľa pomocou mena a hesla. Na výber boli autentizačné protokoly \textit{Challenge Handshake Authentication Protocol} (\acrshort{chap}) \cite{chap}, \textit{MicroSoft Challenge Handshake Authentication Protocol} (\acrshort{mschap}) \cite{mschap} a \textit{Password Authentication Protocol} (\acrshort{pap}) \cite{pap}.  
	\item{\textbf{Kontrolu prístupu}} -- z ang. \textit{Access Control}, po úspešnej autentizácií je následne prístup používateľa riadený na základe pravidiel a politiky prístupu daného OS. 
	\item{\textbf{Šifrovanie dát}} -- z ang. \textit{Data Encryption}, je vykonané pomocou vopred zdieľaného kľúča, ktorý sa získa odvodením z hašovanej hodnoty uloženého hesla používateľa. Haš je vstupom do prúdovej šifry RC4 \cite{rc4} a výstupom je  40-bitový kľúč relácie. 
	\item{\textbf{Filtrovanie PPTP paketov}} -- z ang. \textit{PPTP Packet Filtering}, možnosť zapnutia filtrovania paketov len autentizovaných PPTP klientov.
	\item{\textbf{Preddefinované Firewall pravidla pre PPTP}} -- z ang. \textit{PPTP with Firewalls and Routers}, PPTP má štandardne definovaný TCP port 1723 a ID 47 v IP protokole. Vďaka tomu je možné jednoducho presmerovať tok dát.
\end{itemize} 

Protokol je aktuálne štandardne zahrnutý v každej distribúcií \acrshort{os} Windows aj Linux. Pri vytváraní VPN siete teda používateľ môže zvoliť uvedený protokol na zabezpečený prenos dát v lokálnej, ale aj naprieč verejnou sieťou. Výhodou protokolu je vysoká kompatibilita naprieč rôznymi platformami. Nevýhodou je veľké množstvo zraniteľných miest, možností zneužitia a použitie starších kryptografických algoritmov. Aktuálne existujú protokoly poskytujúce lepšiu bezpečnosť ako uvedená implementácia. Z uvedených dôvodov sa tento protokol neodporúča používať.


\subsection{\acrfull{l2tp}}
Ďalším z VPN tunelovacích protokolov je \acrshort{l2tp}. Špecifikovaný bol v roku 2000 v dokumente \acrshort{rfc}\footnote{z ang. \textit{Request For Comments}} 2661 \cite{rfc2661}. Inšpirovaný dvoma staršími protokolmi \acrshort{l2f}\footnote{z ang. \textit{Cisco Layer 2 Forwarding Protocol}} \cite{rfc2341} a PPTP.   

L2TP sieť pozostáva primárne z 3 zariadení:
\begin{enumerate}
	\item{\textbf{PPP terminál}} -- ľubovoľné zariadenie na vykonanie PPP enkapsulácie na dáta a pripojenie k \acrshort{lac}\footnote{z ang. \textit{L2TP access concentrator}}. Môže to byť aj samotné zariadenie, ktoré sa pripája. 
	\item{\textbf{L2TP prístupového servera}} -- \acrshort{lns}\footnote{z ang. \textit{L2TP Network Server}}, jeden z koncov tunela, ktorý deenkapsuluje dáta a poskytuje prístup do lokálnej sieti. Na zariadení prebieha autentizácia používateľa, nastolenie PPP relácie\footnote{z ang. \textit{session}} a L2TP tunela s \acrshort{lac}. Nasadzuje sa na hranici medzi privátnou a verejnou sieťou, zvyčajne ako sieťová brána na opustenie danej súkromnej siete\footnote{z ang. \textit{gateway}}. Obdobne poskytuje funkcionalitu prekladu adries z privátnych do verejných a opačne. Uvedená funkcionalitá sa nazýva \acrfull{nat}. Viac o tomto protokole je dostupné na \cite{nat}.   
	\item{\textbf{L2TP koncentrátora}} -- \acrshort{lac}, zariadenie umiestnené medzi LNS a clientom. Služí na preposielanie paketov oboma smermi. V smere k LNS vytvára L2TP tunel. LAC server môže byť nasadený aj na PPP termináli a pracovať ako \acrshort{poe}\footnote{z ang. \textit{PPP Over Ethernet}} server. Zvyčajne je koncentrátor nasadený na \acrshort{nas}, ale je možné ho nasadiť na ľubovoľné sieťové zariadenie.
\end{enumerate}
V schéme \ref{l2tp} je znázornené ako sú spomenuté zariadenia zapojené v logickej topológií. Na schéme je znázornený aj proces enkapsulácie PPP paketov, prenosu dát naprieč tunelom a následnej deenkapsulácie. 
% TODO: \usepackage{graphicx} required
\begin{figure}
	\centering
	\includegraphics[width=0.7\textwidth]{figures/l2tp}
	\caption{}
	\label{l2tp}
\end{figure}

L2TP používa namiesto TCP protokolu UDP, ktorého výhodou je rýchlejší prenos bez kontroly prijatia na druhej strane spojenia.  
Pri vzniku tunela používa UDP port 1701. Iniciátor tohto procesu následne vyberá náhodne z nečinných portov a smeruje nim pakety s portom 1701. Prijímač po prijatí paketu taktiež náhodne určí nečinný port a preposiela ním pakety prijaté iniciátorom. Takto zvolené portové čísla sa používajú až kým nie je komunikácia cez tunel ukončená. 

L2TP vytvára 2 druhy spojenia počas vytvárania konektivity medzi LAC a LNS. 
\begin{itemize}
	\item{\textbf{tunelové spojenie}} -- z ang. \textit{tunnel connection}, napomáha k nastoleniu viacero tunelov medzi zariadeniami. Pozostáva z jedného alebo viacerých relačných spojení. Žiadosť o vytvorenie vytvára LAC server po prijatí PPP žiadosť od vzdialeného používateľa. LAC a LNS si vymenia informácie potrebné na vznik spojenia ako sú napríklad autentizačné informácie tunela a ID. Po úspešnom vyjednávaní\footnote{z ang. \textit{negotiation}} vznikne tunel, ktorý je identifikovateľný pomocou dohodnutého ID.  
	\item{\textbf{relačné spojenie}} -- z ang. \textit{session connection}, reprezentuje PPP spojenie naprieč tunelom. Môže vzniknúť až keď je tunel úspešne vytvorený.	
\end{itemize}     
Po vytvorení oboch spojení následne odchádza k prenosu zapuzdrených PPP paketov naprieč týmto tunelom.

L2TP ponúka kompatibilitu pre mnohé platformy a jednoduchú konfiguráciu. OS Windows, Linux a Mac majú tento protokol zabudovaný v sebe. Výhodou je použitie UDP, vďaka tomu je možné protokol používať aj v nestabilnom sieťovom prostredí.
Nevýhodou je zníženie prenosovej rýchlosti. L2TP používa taktiež vopred zdieľané kľúče a v prípade ak sa nezhodujú, tak dochádza k zastaveniu chodu. L2TP je podporuje iba limitovaný počtet portových čísel. Samotná implementácia L2TP neposkytuje, respektíve nezabezpečuje žiadne šifrovanie alebo autentizáciu paketa. Na zabezpečenie sa používa v kombinácií s iným protokolom. Veľmi známa je implementácia L2TP/IPSec. 

V roku 2005 vznikla 3. verzia protokolu, ktorá priniesla zväčšenie dĺžky ID v L2TP hlavičke, zo 16 na 32 bitov. Rozšírenie tunelového autentizačného mechanizmu a oddelenie L2TP dát súvisiacich s PPP protokolom. Viac o tomto upravenom protokole je možné nájsť na \cite{l2tpv3}.
  
Viac informácií o L2TP problematike je možné nájsť v \cite{l2tp}, \cite{rfc2661}, \cite{l2tphuawei}, odkiaľ boli informácie z tejto podkapitole čerpané. 

\subsection{Internet Protocol Security (\acrshort{ipsec})}
\acrshort{ipsec} je otvorený štandard. Vďaka tomu vďačí za veľkú popularitu a pravidelné aktualizácie kryptografických algoritmov. Protokol sa využíva na zabezpečenie bezpečnosti v sieti. IPSec môžeme klasifikujeme ako L3 protokol nakoľko pracuje s dátami zo sieťovej vrstvy. Má dva režimy:
\begin{itemize}
	\item{\textbf{transportný režim}} -- po prijatí paketu z vyššej vrstvy sú smerovacie dáta zachované a na základe nich sú dáta odosielané ďalej. K zvyšným dátam je pridaná hlavička použitého IPSec protokolu. Princíp pridania dát k paketu je znázornený na \ref{iptransport}  
	\item{\textbf{tunelovací režim}} -- zapuzdruje paket z vyššej vrstvy. Pridáva novú IP hlavičku a hlavičku IPSec protokolu k pôvodnému nezapuzdrenému paketu. Uvedená skutočnosť je zobrazená na obrázku \ref{iptunel}
\end{itemize}
% TODO: \usepackage{graphicx} required
\begin{figure} [!h]
	\centering
	\includegraphics[width=0.7\textwidth]{figures/iptransport}
	\caption{IPSec transportný režim}
	\label{iptransport}
\end{figure}
% TODO: \usepackage{graphicx} required
\begin{figure} [!h]
	\centering
	\includegraphics[width=0.7\textwidth]{figures/iptunel}
	\caption{IPSec tunelovací režim}
	\label{iptunel}
\end{figure}


Za účelom poskytnutie zabezpečeného spojenia, protokol vykonáva autentizáciu, šifrovanie a vyjednávanie, resp. výmenu potrebných kľúčov. Jednotlivé činnosti sú realizované pomocou týchto IPSec protokolov:
\begin{itemize}
	\item{\textbf{Autentizačná hlavička}} -- z ang. \textit{Authentication Header}, pridáva k prepravovanému paketu dáta na zabezpečenie dátovej integrity a pôvodu. Chráni proti z ang. \textit{Replay attack} \cite{repa}. Dáta v tomto režime nie sú šifrované. AH hlavička je vygenerovaná v závislosti od toho v akom IPSec režime je protokol použítý.    
	\item{\textbf{Bezpečnostné zapuzdrenie nákladu}} -- z ang. \textit{Encapsulating Security Payloads}, narozdiel od AH, ESP aj šifruje dáta z vyššej vrstvy. Vďaka tomu dochádza k zabezpečeniu dôvernosti, datovej integrity a autentizácie pôvodu. Výsledná hlavička závisí od dát zo sieťovej vrstvy a konfigurácie IPSec.    
	\item{\textbf{\acrlong{isakmp}}} -- ďalej \acrshort{isakmp}, protokol na autentizáciu a výmenu kľúčov. Slúži taktiež na vytvorenie parametra \acrshort{sa}, ktorý sa používa v hlavičke \acrshort{ah}/\acrshort{esp}.
\end{itemize}

IPSec je možné nakonfigurovať, tak aby používal \acrshort{ah} a \acrshort{esp} selektívne alebo aj súčasne. V závislosti od konfigurácie následne dochádza k zapuzdrovaniu prichádzajúceho paketu. Používateľ má na výber viacero štandardných kryptografických algoritmov. Príkladmi sú \acrshort{aes} \cite{aes}, \acrshort{rsa} \cite{rsa}, Diffie-Hellman \cite{dh} a  eliptické krivky (\acrshort{ecdsa} \cite{ecdsa} aj \acrshort{ecdh} \cite{ecdh}).

Nabudúce sa opytat ci biks predmet moze byt zdroj alebo ine. \cite{biks}

\subsection{Secure Socket Tunneling Protocol (\acrshort{sstp})}
SSTP je bežný L2 VPN protokol, ktorý zapuzdruje PPP ramce cez \acrshort{https} protokol \cite{https}. Spolieha sa na \acrshort{ssl}, resp. \acrshort{tls}, ktoré sú opísané v úvode práce. Vďaka tomu umožňuje ľahší prechod cez väčšinu firewallov a proxy brán. Teda blokovanie takto vytvoreného VPN spojenia je pre poskytovateľov internetu a správcov siete zložitejšie. 

SSTP bol vytvorený v roku 2007 spoločnosťou Microsoft. Primárne pre platformu Windows. Cieľom bolo poskytnúť bezpečnejšiu náhradu za PPTP a L2TP. V súčasnosti sa považuje za jeden zo štandartných protokolov. Je dostupný vo viacerých operačných systémoch vrátane Linuxu a BSD. Pravidelne udržiavaný o čom svedčia aj priebežná aktualizácia dokumentácie na Microsoft dokumentačných stránkach. Informácie k tejto podkapitole boli čerpané z \cite{mssstp}. V uvedenom zdroji je možné nájsť podrobnejšie informácie o protokole. 

Proces enkapsulácie s využitím protokolov je znázornený pomocou schémy \ref{sstpprotocolstack}. 
% TODO: \usepackage{graphicx} required
\begin{figure}[!h]
	\centering
	\includegraphics[width=0.3\textwidth]{figures/sstpprotocolstack}
	\caption{Proces enkapsulácie PPP rámcov pomocou protokolov naprieč SSTP protokolom}
	\label{sstpprotocolstack}
\end{figure}
Pri vytváraní SSTP segmentu dochádza k zapuzdreniu PPP rámcov pomocou HTTPS s využitím TCP protokolu s portovým číslom 443. Po úspešnom nadviazaní TCP a overení SSL/TLS spojenia dochádza k spracovaniu SSTP hlavičky. Po úspešnom odstránení sa následne získa prístup k pôvodnému PPP rámcu.   

Podobne ako tomu bolo v prípade L2TP protokolu, opísaného vyššie, tak z hľadiska premávky putujú naprieč tunelom dva druhy paketov. Vytvára a odosiela ich klient aj server. Majú špecifický formát a musia byť prenášané po bajtoch a v sieťovom poradí bajtov\footnote{z ang. \textit{network byte order}}, z ľava do prava, teda od najvýznamnejšieho bitu po najmenej významný\footnote{sieťové spracovanie dát známe aj ako z ang. \textit{Big Endian}}. Prvé 4 parametre v oboch hlavičkách sú rovnaké: 
\begin{enumerate}
	\item{\textbf{Verzia}} -- z ang. \textit{Version}, má veľkosť 8-bitov, používa sa pri komunikácií a vyjednávaní SSTP verzie, ktorá sa má použiť. Prvé 4 bity signalizujú majoritnú verziu a zvyšné minoritnú verziu. 
	\item{\textbf{Reservované}} -- z ang. \textit{Reserved}, 7 bitov nastavených na 0, rezervované pre budúce použitie, pri spracovaní sa ignorujú.
	\item{\textbf{C}} -- 1 bit, indikátor pre dátový (0) a kontrolný SSTP paket (1). 
	\item{\textbf{Dĺžka paketu}} -- z ang. \textit{LengthPacket}, 16-bitový parameter, pozostáva z:
		\begin{itemize}
			\item{\textbf{R}} -- 4 bity, pripravené na budúce použitie, nastavené na 0 a ignorované,
			\item{\textbf{Dĺžka}} -- z ang. \textit{Length}, 12 bitov, špecifikuje bajtovú veľkosť celého SSTP paketu
		\end{itemize}
\end{enumerate}
 
Rozdielnosť medzi hlavičkami vzniká v:
\begin{itemize}
	\item{\textbf{Dátové SSTP pakety}} -- z ang. \textit{SSTP Data Packets} \cite{datpak}, obsahujú 5. parameter 
		\begin{itemize}
			\item{\textbf{5. Dáta}} -- pole variabilnej dĺžky. Obsahuje zapuzdrený náklad z vyššej vrstvy.
		\end{itemize} 
	\item{\textbf{Kontrolné SSTP pakety}} -- z ang. \textit{SSTP Control Packets} \cite{conpak}, obsahuje 3 dodatočné parametre:
		\begin{itemize}
			\item{\textbf{5. Typ správy}} -- z ang. \textit{Message Type}, 16-bitové pole s SSTP správou o stave spojenia. Celkovo 9 možných správ \cite{conpak}. 
			\item{\textbf{6. Číselné atribúty}} -- z ang. \textit{Num Attributes}, 16-bitové pole, ktoré špecifikuje počet parametrov v správe
			\item{\textbf{7. Atribúty}} -- z ang. \textit{Atributes}, zoznam parametrov s variabilnou veľkosťou.
		\end{itemize}	
\end{itemize}   
 
Protokol vo všeobecnosti nepodporuje spojenie typu sieť k sieti. Primárne je orientovaný na pripojenie klienta k sieti, za účelom získania vzdialeného prístupu. To isté platí pri autentizácií. Je možné autentizovať iba používateľa, iné možnosti nie sú podporované (zariadenie, smart card, počítať,...). V prípade nestabilného spojenia, je možný vysoký výskyt straty paketov. Dôvodom je použitie TCP protokolu. 

Bezpečnosť SSTP je sprostredkovaná za pomoci HTTPS protokolu, ktorý používa SSL/TLS. Aplikované kryptografické algoritmy závisia od verzie SSL/TLS, ktorá je použitá v implementácií. TLS bol predmetom opisu v úvode práce. SSTP sa považuje za veľmi bezpečný protokol. Na druhej strane použitie robustných šifrovacích a autentizačných algoritmov, dosť spomaľuje výsledné SSTP VPN pripojenie.

Viac informácií o protokole môže čitateľ nájsť v \cite{mssstp}, odkiaľ boli aj informácie pre túto podkapitolu čerpané.  


\subsection{Ostatné populárne VPN protokoly}
Vrámci tejto práce si predstavíme ešte dvojicu veľmi populárnych VPN riešení. Koncepčne sú riešenia pri vytváraní tunelu a prenose správ podobné vyššie opísaným protokolom. Z uvedeného dôvodu protokoly len stručne charakterizujeme.
\subsubsection{OpenVPN}
OpenVPN je jeden z najznámnejšich voľne dostupných VPN tunelovacích protokolov. Vznikol v roku 2001. Dostupné ako firmvérové riešenie pre výrobcov sieťových zariadení. Pre používateľa je dostupná vo forme softvéru, ktorý je potrebné nainštalovať na cieľovú platformu. Ponúka aj grafické rozhranie. Zabezpečené spojenie naprieč internetom je sprostredkovane prostredníctvom SSL/TLS protokolu. Dokáže pracovať s dátami na L2 aj L3 vrstve v závislosti od konfigurácie. Vďaka voľnému prístupu do zdrojového kódu poskytuje jednoduché možnosti na skúmanie výsledných riešení, validáciu a prípadnú úpravu podľa potrieb konkrétneho používateľa. Protokol je napísaný v jazyku C. Bezpečnostné prvky su implementované pomocou OpenSSL knižnice. Knižnica v sebe zahŕňa funkcie potrebné na šifrovanie, autentizáciu, výmenu kľúčov a mnoho ďalšieho. Na obrázku \ref{ovpnptstrc} je znázornená výstupna štruktúra paketu po zapuzdrení. 
% TODO: \usepackage{graphicx} required
\begin{figure}[!h]
	\centering
	\includegraphics[width=0.5\textwidth]{figures/ovpnptstrc}
	\caption{}
	\label{ovpnptstrc}
\end{figure}

Aktuálne používa na šifrovanie AES s 256-bitovou veľkosťou kľúča. V súčasnosti sa OpenVPN považuje za najbezpečnejšiu implementáciu VPN dostupnú pre viacero platforiem. Používateľ ma možnosti vo voľbe protokolu TCP alebo UDP. Podporuje aj prácu s IPv6. OpenVPN nie je kompatabilná s ostatnými protokolmi. Je preto potrebné mať implementáciu na serveri aj klientovi. Nevýhodou je pomerne veľké, energeticky a výpočtovo náročné prevedenie.

Viac informácií o tomto komplexnom programe je možné nájsť v \cite{vpntech}, \cite{ovpn} a priamo na stránke OpenVPN\footnote{https://openvpn.net/faq/what-is-openvpn/}. Z uvedených zdrojov boli aj informácie čerpané. Zdrojový kód je dostupný napríklad na Githube\footnote{https://github.com/OpenVPN/openvpn/}. 
\subsubsection{WireGuard}
WireGuard je najnovší protokol z vyššie uvedených. Pracuje iba na L3 vrstve. Jedná sa o implementáciu s voľne dostupným zdrojovým kódom. Vznikol v roku 2015. Cieľom projektu bolo vytvoriť jednoduchý protokol, ktorý sa ľahko používa, dosahuje vysoké prenosové rýchlosti a poskytuje kvalitnú bezpečnosť pre používateľa. Vo výsledku sa to vývojárom podarilo. Od roku 2020 sa stal súčasťou Linuxového jadra, konkrétne verzie Linux 5.6 kernel. Aktuálne podporuje veľké množstvo OS vrátane Androidu, Windowsu, MacOS, OS založené na BSD \cite{bsd} a ďalšie. Za úspešne prevedenie vďačí WireGuard implemetovaniu svojich funkcionalít do samotných jadier OS, čo značne zrýchľuje spracovanie dát. Samozrejmostou ostáva aj využitie moderných, extrémne rýchlych kryptografických algoritmov a podpora pre IPv4 a IPv6. 

Protokoly použité vo WireGuarde sú:
\begin{itemize}
	\item{\textbf{X25519}} \cite{x25519} -- zabezpečuje výmenu kľúčov vďaka kryptografii s eliptickým krivkami \cite{ecc} (ďalej \acrshort{ecc}), ponúka 128-bitovú bezpečnosť s veľkosťou kľúča 256-bitov. Považuje sa za jednu z najrýchlejších kriviek v \acrshort{ecc}.
	\item{\textbf{ChaCha20}} \cite{chacha} -- zabezpečuje symetrické šífrovanie.
	\item{\textbf{Poly1305}} \cite{poly} -- vytvára autentifikačný kód správy\footnote{z ang. \textins{Message Authentication codes}}. Veľmi častá je kombinácia ChaCha20-Poly1305 za účelom autentizovaného šifrovania s pridruženými dátami\footnote{z ang. \textit{Authenticated encryption with associated data}}.
	\item{\textbf{SipHash}} \cite{siphash} -- algoritmus sa používa na vytvorenie a mapovanie kľúčov k hodnotám v hashovacej tabuľke. Hashovacie tabuľky sú známy pojem v oblasti dátových štruktúr. Ich hlavnou výhodou je vysoká rýchlosť v porovnaní s ostatnými možnosťami.  
	\item{\textbf{BLAKE2s}} \cite{blake} -- hashovacia funkcia, rýchlejšia než aktuálne štandardy z rodiny SHA. 
	\item{\textbf{UDP}} --  protokol na prepravu zapuzdrených dát.
\end{itemize} 
WireGuard podporuje aj použitie vopred zdieľaného kľúča za účelom symetrického šifrovania. Dôvodom je pokrok v oblasti kvantových počítačov, ktoré predstavujú riziko pre algoritmy založené na asymetrickom šifrovaní.

Viac informácií o tomto protokole je k dispozícií na \cite{wireguard} a \cite{wireguardpdf}, odkiaľ boli informácie aj čerpané. Zdrojový kód je rozdelený do viacerých repozitárov. Zoznam je dostupný na stránkach Wireguardu\footnote{https://www.wireguard.com/repositories/}.
\section{Zhrnutie VPN sieti}
Jednoducha schema so vsetkym v jednom obrazku \\

 
\cite{divvpn}, \cite{ciscovpn}
https://www.vpnmentor.com/blog/different-types-of-vpns-and-when-to-use-them/
https://www.top10vpn.com/what-is-a-vpn/vpn-types/
%https://en.wikipedia.org/wiki/Virtual_private_network
%https://csrc.nist.gov/glossary/term/virtual_private_network
https://www.auvik.com/franklyit/blog/vpn-types/
https://www.geeksforgeeks.org/types-of-virtual-private-network-vpn-and-its-protocols/?ref=rp
https://www.geeksforgeeks.org/difference-between-site-to-site-vpn-and-remote-access-vpn/?ref=rp
\\
V~súčasnosti je trend, využívanie VPN na súkromné účely. Dôvodov môže byť viacero. Napríklad prístup k lokálne blokovaným doménam, anonymita v internetovom prostredí, zabezpečenie pripojenia a iné. V tomto prípade vstupujú do popredia tzv. VPN poskytovatelia (z ang. \textit{VPN Providers}). Ktorý za poplatok poskytujú výhody pripojenia cez VPN k verejnej sieti.
-- stoji za zmienku \\
https://github.com/vpnhood/VpnHood a plus v inych jazykoch 

\chapter{Ľahká kryptografia}\label{krypto}
https://csrc.nist.gov/projects/lightweight-cryptography/finalists 
\\https://csrc.nist.gov/CSRC/media/Projects/Lightweight-Cryptography/documents/final-lwc-submission-requirements-august2018.pdf

V počítačovej sfére sa do popredia dostávajú zariadenia, ktorých hardvérové prostriedky a ponúkaný výkon sú podstatne nižšie ako v prípade bežne dostupných zariadení na domáce, resp. komerčné použitie. Príkladom môžu byť \acrshort{iot} zariadenia, senzorové uzly, mikro-ovládače a podobné. Dôležitou vlastnosťou je aj komunikácia medzi sebou alebo inými zariadeniami. V dôsledku toho je nutné riešiť aj otázku zabezpečenia, resp. bezpečnosti takého prenosu dát. 

Kryptografia je oblasť počítačovej vedy, ktorá sa zaoberá práve spomenutou problematikou. Hlavným cieľom je utajiť správu pri jej ceste z bodu A do B. Teda od odosielateľa (tvorcu) správy, až k jej prijímateľovi. Dôsledkom tohto úkonu dochádza k zabezpečeniu 3 hlavných úloh kryptografie:
\begin{itemize}
	\item \textbf{ochrana osobných údajov} (dôvernosť) -- z ang. \textit{Data Privacy}, 
	\item \textbf{autenticita údajov} (prišla z miesta, kde sa uvádza)  -- z ang. \textit{Data Authenticity},
	\item \textbf{integrita údajov} (nebolia upravená počas prenosu)  -- z ang. \textit{Data Integrity}.
\end{itemize} 
Pojem kryptografia vznikol už pomerne dávno a bol už viackrát charakterizovaný. Z toho dôvodu sa tejto problematike ďalej nebudeme venovať. Ak by však bol čitateľ v tejto oblasti nový, odporúčame najprv načerpať viac informácií. Kvalitné spracovanie problematiky je možné nájsť v \cite{levicky}. 

V súčasnosti má používateľ možnosť vybrať si zo širokej ponuky kryptografických algoritmov. Výber je volený na základe potrebnej funkcionality, ktorú sa snažíme implementovať. Napríklad za účelom symetrického šifrovania by sme mohli použiť \acrshort{aes} \cite{aes}, ktorý je aktuálne používaný ako štandardný kryptografický algoritmus. Detailný opis jednotlivých blokov a postupov použitých v AES-e, je obsahom rôznych vydaní. Viac informácií o problematike nájde čitateľ v \cite{levicky}.

V prípade zariadení spomenutých v úvode kapitoly, však môže nastať problém s výpočtovým výkonom pri realizácií algoritmov. Vo výsledku trvá vykonávanie funkcionality omnoho dlhšie ako v prípade normálnych zariadení. V roku 2005 bol prvýkrát definovaný pojem, z ang. \textit{Lightweight Cryptography}. Konkrétne v práci \cite{lwc}. Vrámci tejto práce voľne preložíme tento pojem ako \textbf{Ľahká kryptografia} (ďalej \acrshort{lwc}). 

\acrshort{lwc} algoritmy sú mnoho násobne efektívnejšie ako súčasne používané konvenčné kryptografické štandardy. Dôvodom je ich vysoká rýchlosť a nízky počet potrebných procesorových inštrukcií na vykonanie cielenej funkcionality. V súčasnosti je častým javom použitie označenia \textit{lightweight} pre ľubovoľný kryptografický algoritmus. Jedná sa o implementácie, ktoré svojimi vlastnosťami spĺňajú základné požiadavky LWC. Tie boli definované samotnými autormi už v diele \cite{lwc}. Sú nimi: 
\begin{itemize}
	\item{\textbf{minimalizácia spotreby zdrojov zariadenia}} -- veľkosť kódu, používaných dát a spotreby energie,
	\item{\textbf{poskytnutie vysokého stupňa bezpečnosti}},
	\item{\textbf{odolnosť voči útokom tzv. z ang. \textit{side-attack}}} -- napríklad útokom na analýzu výkonu a na časovanie, 
	\item{\textbf{jednoduchá implementácia a efektívnosť,}}
	\item{\textbf{nízka pamäťová stopa}} -- z ang. \textit{low memory footprint}.
\end{itemize}

V práci budeme tieto algoritmy označovať ako \textit{Lightweight Cryptographic Algorithm} (ďalej \acrshort{lwca}).
\acrshort{lwca} tvoria kryptografické algoritmy, ktoré splňajú vyššie stanovené vlastnosti a teda je možné ich aj nasadiť do tzv. \textit{low resource} zariadení. V závislosti od výslednej implementácie sa sledujú požiadavky na daný algoritmus. V prípade hardvérovej implementácie najväčšiu rolu pri tvorbe optimálneho algoritmu zohráva energetická spotreba a potrebná veľkosť čipu (chip size iba v zdroji -- In hardware implementations, chip size and/or energy
consumption are the important measures to evaluate the lightweight properties.). Na druhej strane, pri softvérovej implementácií je dôležitým aspektom veľkosť algoritmu a/alebo jeho využitie RAM pamäte. Čím sú uvedené parametre nižšie, tým výhodnejší je nasadenie daného algoritmu pre cieľové zariadenie. V prípade slabých zariadení sa pri výslednej implementácií \acrshort{lwca} môžeme stretnúť aj s kompromisom medzi ponúknutou bezpečnosťou a efektivitou.

V rámci tejto kapitoly si predstavíme jeden z novších \acrshort{lwca} algoritmov. Konkrétne balíček Xoodyak s kryptografickou permutáciou XOODOO \cite{tkecak} a jeho použitie. Xoodyak sa stal jedným z 10 finalistov v \acrshort{lwc} štandardizačnom procese Národného inštitútu pre štandardy a technológie (ďalej \acrshort{nist}) \cite{lwc3}. Aktuálne ešte stále prebieha výber štandardu.

Viac podrobností nájde čitateľ v \cite{lwc}, \cite{lwc2} a \cite{lwc3}, odkiaľ boli aj informácie z tejto kapitoly čerpané. 
        
\section{Kryptografická permutácia XOODOO a jej variácie}
Permutácia je operácia, ktorá mení pozíciu vstupných prvkov, tak aby vo výsledku vzniklo nová usporiadaná množina. Kryptografické permutácie sú špeciálne navrhnuté matematické algoritmy, tak aby bolo možné využiť ich za účelom šifrovania a dešifrovania. Operácie vykonané v takýchto permutáciách musia byť invertibilné. Kryptografické permutácie tvoria základný stavebný blok pri následnej tvorbe ďalších kryptografických blokov. Množina základných stavebných blokov sa v kryptografii spoločne označuje ako kryptografické primitíva (ďalej \acrshort{kp}). Príkladom môžu byť hašovacie funkcie, generátory náhodných čísel a podobne. Viac informácií nájde čitateľ v \cite{kp}.  
 
XOODOO je sada 384-bitových kryptografických permutácií parametrizovaných počtom kôl. Funkcia kola/rundy\footnote{z ang. \textit{round}} funguje na 12 slovách\footnote{z ang. \textit{words}} po 32 bitoch. Vďaka tomu je efektívna aj na menej výkonných procesoroch nižšej triedy. Vytvoril ju tím Keccak, ktorý stojí za viacerými úspešnými kryptografickými algoritmami. Napríklad hashovacie funkcie z rodiny SHA-3 a iné -- \cite{kecsup}. XOODOO algoritmus vznikol po vytvorení tzv. Kravatte autentizačno-šifrovacieho algoritmu \cite{kravatte}, založené na Keccak-p permutácií \cite{keccakp}. Ten sa ukázal ako dostatočne rýchly na širokom spektre platforiem. Avšak nezapadá do kategórie \acrshort{lwca}.

Tím Keccak vypracoval nové riešenie. Ním bol port medzi ich prvotným Keccak-p dizajnom a Gimli-ho \cite{bernstein2017gimli} permutačným algoritmom. Vo výsledku autori zlúčili lepšie realizované prvky z oboch algoritmov do jedného celku. Primárny problém samotnej Gimli permutácie bol v slabom prejave zmeny výstupu po malých zmenách vo vstupnej správe. Táto vlastnosť sa v kryptografii označuje pomocou anglického pojmu, tzv. \textit{propagation properties}\footnote{Cieľom je aby aj zmena jedného bitu na vstupe, ovplyvnila čo najviac bitov vo výstupe -- tzv. \textit{Lavínový efekt}}. Novo-vzniknuté riešenie autori pomenovali XOODOO. Na základe rôznych variácií tohto kryptografického primitíva sa im následne podarilo vytvoriť sadu vysoko efektívnych kryptografických funkcií. 
Medzi sady, ktorých jadro tvorí XOODOO, patrí Xoodyak a Xoofff. Xoofff pozostáva zo zlúčenia Farfalle konštrukcie \cite{farfalle} so XOODOO permutáciou. 

Xoodyak ma narozdiel od Xoofff duplexovú konštrukciu \cite{duplex}. Vo výsledku máme ľahko prenosnú, všestrannú, kryptografickú knižnicu. Je vhodná do výkonovo obmedzených prostredí. Môže sa použiť pre väčšinu kryptografických funkcií, ktoré používajú symetrický kľúč. Napríklad hashovanie, šifrovanie, výpočet MAC alebo autentizované šifrovanie. O kvalite riešenia napovedá aj fakt, že sada Xoodyak je jedným z 10 finalistov v oblasti ľahkej kryptografie NIST štandardizačného procesu.
V rámci tejto kapitoly opíšeme kryptografické primitívum XOODOO a následne balík Xoodyak. Informácie o téme boli čerpané z týchto zdrojov: \cite{tkecak}, \cite{xd}, \cite{xcb}, \cite{xoodoocb}, \cite{xdupdate},\cite{xdr1}.
\subsection{XOODOO permutácia}
XOODOO je permutácia, definovaná počtom rúnd. Má klasickú iteračnú štruktúru. Teda opakovane sa vola rundová funkcia s aktuálnym stavom. Pre pochopenie operácií je nutné pochopiť určité označenie použité v algoritme.

Stav -- \textbf{state}, pozostáva z 3 rovnako veľkých horizontálnych rovín -- \textbf{planes}\footnote{V jednej rovine je 128 bitov}. Každá z týchto rovín obsahuje štyri paralelne 32-bitové pruhy -- \textbf{lanes})\footnote{V jednom pruhu je 32 bitov}. Okrem tejto charakteristiky je možné opísať stav ako množinu zloženú zo stĺpcov -- \textbf{columns}\footnote{V jednom stĺpci je 12 bitov}, pričom jeden stĺpec obsahuje 4 bity v jednej rovine. Stav je teda tvorený zo stĺpcov usporiadaných v poli o rozmere $4\times3\times32 = 384$ bitov. Posledná položka na opis stavu sú tzv. listy -- \textbf{sheets}\footnote{V jednom liste je 96 bitov}. List sa skladá z 3 na sebe uložených pruhov. Uvedené pojmy sú znázornené pomocou schémy \ref{xoodooterm}, ktorá bola prebraná z \cite{xcb}.

\begin{figure}[!h]
	\centering
	\includegraphics[width=1\textwidth]{figures/xoodooTerminology}
	\caption{Grafické znázornenie terminológie využitej v kryptografickej permutácii XOODOO \cite{xcb}}
	\label{xoodooterm}
\end{figure}

Roviny majú index $y$. Index $y=0$ zodpovedá spodnej rovine a vrchná rovina má index $y=2$. Bit je označený s indexom $z$ vrámci množiny pruhov. List označujeme pomocou indexu $x$. Pozícia pruhu v stave je definovaná pomocou dvoch súradníc $(x,y)$. Konkrétny bit je možné reprezentovať v stave pomocou trojice súradníc $(x,y,z)$. Pri určení stĺpca sú potrebné 2 súradnice $(x,z)$. Pred spustením samotného algoritmu musí používateľ vykonať mapovanie 384-bitovej správy voči horizontálnym rovinám. Tento úkon sa realizuje pomocou vzorca \ref{index}.

\begin{equation}\label{index}
	i=z+32(x+4y)
\end{equation}
Výhodou XOODOO je, že celý stav, 384 bitov, dokáže byť uložený v 12 registroch po 32 bitov. Vďaka tomu ideálne vyhovuje nízko výkonným 32-bitovým zariadeniam.

Rundová funkcia pozostáva z 5 krokov:
\begin{enumerate}
	\item miešanie vrstvy -- z ang. \textit{a mixing layer $\theta$}, 
	\item posun rovín -- z ang. \textit{a plane shifting $\rho_{west}$}, 
	\item pridanie rundových konštánt -- z ang. \textit{the addition of round constants $\iota$},
	\item nelineárna vrstva -- z ang. \textit{a non-linear layer $\chi$},
	\item posun rovín -- z ang. \textit{an another plane shifting $\rho_{east}$}.
\end{enumerate}
Opis jednotlivých krokov je znázornený pomocou schémy \ref{xoodoochi}, \ref{xoodooml}, \ref{xoodooshift} ktoré sú prebraté z \cite{xcb}.  

% TODO: \usepackage{graphicx} required
\begin{figure}[h!]
	\centering
	\includegraphics[width=0.5\textwidth]{figures/xoodoochi}
	\caption{Grafické znázornenie operácie $\chi$ \cite{xcb}}
	\label{xoodoochi}
\end{figure}
% TODO: \usepackage{graphicx} required
\begin{figure}[h!]
	\centering
	\includegraphics[width=0.5\textwidth]{figures/xoodooml}
	\caption{Grafické znázornenie operácie $\rho$ \cite{xcb}}
	\label{xoodooml}
\end{figure}
% TODO: \usepackage{graphicx} required
\begin{figure}[h!]
	\centering
	\includegraphics[width=0.8\linewidth]{figures/xoodooshift}
	\caption{Ilustrácia miešania vrstiev $\rho_{west}$ (vľavo) a $\rho_{east}$ (vpravo)\cite{xcb}}
	\label{xoodooshift}
\end{figure}
K spomenutej schéme \ref{xoodooshift}, by sme rád doplnili, že posun je realizovaný na každom bite. Na schéme sú ilustrované len posuny 2 bitov. 
   
Tabuľka \ref{tab1} vysvetľuje jednotlivé operácie, ktoré sa v algoritme používajú. Algoritmus permutácie je následne znázornený pomocoou \ref{xoodooalgo}. Súčasťou implementácie sú aj tak zvané rundové konštanty. Tie je možné vidieť v \ref{tab2}. Uvedené ilustrácie boli prebrané z dokumentu \cite{xcb}.

\begin{figure}[!h]
	\centering
	\includegraphics[width=1.0\textwidth]{figures/tab1}
	\caption{Charakteristika operácií v algoritmickom zápise kryptografickej permutácie XOODOO \cite{xcb}}
	\label{tab1}
\end{figure}

\begin{figure}[h!]
  	\centering
  	\includegraphics[width=1.0\textwidth]{figures/xoodooalgo}
  	\caption{Algoritmický zápis kryptografickej permutácie XOODOO \cite{xcb}}
  	\label{xoodooalgo}
\end{figure}

\begin{figure}[h!]
	\centering
	\includegraphics[width=1.0\textwidth]{figures/tab2}
	\caption{Súbor rundových konštánt kryptografického algoritmu XOODOO \cite{xcb}}
	\label{tab2}
\end{figure}

\subsection{Kryptograficý balíček Xoodyak} 
Xoodyak možno považovať za všestranný kryptografický nástroj. Je vhodný pre väčšinu operácií využívajúcich symetrický kľúč. Napríklad generovanie pseudonáhodných bitov, autentizáciu, šifrovanie a iné. Tím Keccak použil pri návrhu duplexnú konštrukciu. Konkrétne variant s plným stavom a využitím kľúča. Tento dizajn označujeme ako \acrfull{fskd}. Viac o tejto konštrukcii si čitateľ môže prečítať v \cite{duplex}.
Operačný režim, v ktorom Xoodyak pracuje sa nazýva Cyklista -- z ang. \textit{Cyclist}. Tento názov získal ako opozitum k pomenovaniu režimu Motorista, ktorý je možné nájsť v Keyak schéme \cite{keyak}. Narozdiel od uvedeného balíka Keyak, nie je Xoodyak limitovaný len na autentizované šifrovanie. Je jednoduchší hlavne kvôli tomu že neobsahuje paralelné varianty.

\subsubsection{Režim Cyklista}\label{cyklista}
Režim Cyklista funguje na princípe kryptografických permutácií $f$, teda zmeny usporiadania bitov za pomoci tajného kľúča a matematických operácií. Parametrami sú veľkosti blokov $R_{hash}$, $R_{kin}$, $R_{kout}$ a veľkosť račety, resp. západky\footnote{z ang. \textit{the ratchet size}} \cite{ratchet} $\ell_{ratchet}$. Uvedený pojem sa v kryptografii používa vo forme obrazného pomenovania. Cieľom je poukázať na jednoduchý pohyb vpred, ale s ťažkým, resp. zložitejším pohybom naspäť. Dôležité je, že uvedený scenár je vyvolaný zámerným dizajnom. Šírka permutácie $b'$ je definovaná pomocou vzorca \ref{index2}. Všetky uvedené parametre sú v bajtoch. Pre označenie prázdneho slova budeme používať $E$.
\begin{equation}\label{index2}
	max(R_{hash}, R_{kin}, R_{kout}) + 2 \leq b'
\end{equation} 

Cyklista operuje v dvoch režimoch -- \textbf{hašovací a kľúčový}\footnote{z ang. \textit{hash and keyed mode}.}. 
Inicializácia prebieha pomocou príkazu \lstinline|CYCLIST(K,id,counter)|. Ak sa parameter $K$ rovná prázdnemu slovu $E$, tak potom nastane spustenie v hašovacom režime. Aktuálne nie je do implementácie zakomponovaná možnosť zmeny režimu po inicializácií. Vývojári však túto vlastnosť nevylúčili pre prípadné aktualizácie balíka.  


Dostupné funkcie závisia od režimu, v ktorom sa  Cyklista spúšťa. Medzi ne patria \lstinline|ABSORB()| a \lstinline|SQUEEZE()|. Možno ich volať v oboch režimoch, zatiaľ čo funkcie \lstinline|ENCRYPT()|, \lstinline|DECRYPT()|, \lstinline|SQUEEZEKEY()| a \lstinline|RATCHET()| sú dostupné len pre kľúčový režim. Účel každej funkcie je nasledujúci:
\begin{itemize}
	\item \lstinline|ABSORB(X)| absorbuje vstupný reťazec X,
	\item C $\gets$ \lstinline|ENCRYPT(P)| zašifruje P do C a absorbuje P,
	\item P $\gets$ \lstinline|DECRYPT(C)| dešifruje C do P a absorbuje P,
	\item Y $\gets$	\lstinline|SQUEEZE(L)|  vytvára L-bajtový výstup, ktorý závisí od doteraz absorbovaných dát,
	\item Y $\gets$	\lstinline|SQUEEZEKEY(L)| funguje ako \lstinline|SQUEEZE(L)|, ale používa sa za účelom generovania odvodeného kľúča,
	\item \lstinline|RATCHET()| transformuje stav na nevratný tak, aby sa zabezpečila dopredná bezpečnosť\footnote{z ang. \textit{Forward secrecy}} \cite{fsec}. 
\end{itemize}

Stav bude závisieť od postupnosti volaní funkcií a od jeho vstupných reťazcov. Presnejšie povedané, zámerom je, že akýkoľvek výstup závisí od postupnosti všetkých vstupných reťazcov a volaní, tak že akékoľvek dva nasledujúce výstupné reťazce budú výstupom rôznych domén. Napríklad volanie \lstinline|ABSORB(X)| znamená, že výstup bude závisieť od reťazca $X$. Na druhej strane \lstinline|ABSORB()| vo funkcii \lstinline|ENCRYPT(P)| vytvorí výstup závislý aj od $P$ z funkcie šifrovania. Okrem uvedených závislostí ovplyvňujú výstup aj iné dizajnové riešenia. Príkladom je minimalizácia pamäťovej stopy. Vo výsledku teda výstup závisí od počtu predchádzajúcich volaní funkcie \lstinline|SQUEEZE()| a predtým spracovaných textov pomocou funkcií \lstinline|ENCRYPT()| a \lstinline|DECRYPT()|. Viac informácií o režime je dostupných v kapitole 7.2, publikácie \cite{xcb}.
Algoritmický zápis jednotlivých funkcií a doplňujúce informácie o režime Cyklista je možné nájsť v \cite{xdr2}, konkrétne v kapitole 2.2. 

\subsubsection{Definícia a bezpečnosť}
Xoodyak je definovaný pomocou operatívneho režimu Cyklista nasledovne:
\begin{equation}
	CYCLIST[f,R_{hash},R_{kin},R_{kout},L_{ratchet}]
\end{equation} 
Kde jednotlivé parametre majú veľkosti:
\begin{enumerate}
	\item $f$ -- permutácia XOODOO so šírkou 48 bajtov (384 bitov),
	\item $R_{hash}$ -- 16 bajtov,
	\item $R_{kin}$ -- 44 bajtov,
	\item $R_{kout}$ -- 24 bajtov,
	\item $L_{ratchet}$ -- 16 bajtov.  
\end{enumerate}
Takto definované parametre algoritmu dokážu poskytnúť 128-bitovú bezpečnosť v oboch režimoch Cyklistu. Samozrejmosťou je, že v prípade kľúčového režimu, musí byť veľkosť kľúča rovná alebo väčšia ako 128 bitov. Viac informácií o kryptografickej bezpečnosti algoritmov je možné nájsť v \cite{sec}.

Viac informácií o bezpečnosti Xoodyak-a je možné nájsť v \cite{xcb, 7.3} a \cite{xdr2}, odkiaľ boli informácie čerpané.

\subsection{Možnosti použitia Xoodyak algoritmu}
Obsahom tejto podkapitoly sú uvedené postupy ako a za akých okolností je daný balík možné použiť. 
\subsubsection{Použitie hašovacieho režimu}
Xoodyak sa dá aplikovať ako hašovacia funkcia. Konkrétne je možné ju použiť ako funkciu na rozšírenie výstupu\footnote{z ang. \textit{eXtendable-Output Function}} (ďalej \acrshort{xof}). Nominálne, resp. nie bežné použitie by v tomto prípade bolo nasledujúce: 
\begin{enumerate}
	\item \lstinline|CYCLIST(E,E,E)| -- spustenie v hašovacom režime,
 	\item \lstinline|ABSORB(X)| -- absorpcia vstupného reťazca X,
 	\item \lstinline|SQUEEZE(L)| -- vytvára L-bajtový výstup, ktorý závisí od doteraz absorbovaných dát,
\end{enumerate}
V tomto prípade by sa kryptografická bezpečnosť algoritmu pohybovala v závislosti od veľkosti výstupu L. Konkrétne v intervaloch:
\begin{itemize}
	\item{\textbf{odolnosť voči kolízií}} -- z ang. \textit{collision resistance} \cite{cr}, min(8L/2, 128) bitov,
	\item{\textbf{odolnosť voči }} -- z ang. \textit{preimage and second preimage resistance} \cite{pa}, min(8L, 128) bitov,
	\item{\textbf{odolnosť voči}} -- z ang. \textit{m-target preimage resistance} \cite{pa}, min(8L - log m, 128) bitov.
\end{itemize}
Bežná je však absorpcia sekvencie viacerých reťazcov.

\subsubsection{Použitie kľúčového režimu}
Inicializácia režimu začína použitím príkazu \lstinline|CYCLIST(K,id,counter)|. Autori uviedli celkovo 7 spôsobov využitia. V nich sa opisuje aj význam použitia id a counter parametrov.
\begin{itemize}
	\item{\textbf{Dvoj-cestné zabránenie viac-cieľového útoku}} -- z ang. \textit{Two ways to counteract multi-target attacks},
	\item{\textbf{Tri spôsoby spracovania jednorázového kľúča}} -- z ang. \textit{Three ways to handle the nonce},
	\item{\textbf{Autentizované šifrovanie}} -- z ang. \textit{Authenticated encryption},
	\item{\textbf{Relačné autentizované šifrovanie}} -- z ang. \textit{Session authenticated encryption},
	\item{\textbf{Račetové použitie}} -- z ang. \textit{Ratchet},
	\item{\textbf{Subkľúčové rolovanie}} -- z ang. \textit{Rolling subkeys},
	\item{\textbf{Prepoužitie jednorázového kľúča a uvoľnenie neoverenej dešifrovanej šifry}} -- z ang. \textit{Nonce reuse and release of unverified decrypted ciphertext}.
\end{itemize}

\textbf{Dvoj-cestné zabránenie viac-cieľového útoku} -- id parameter je voliteľný identifikátor kľúča K. Pre každý tajný kľúč by mal byť jedinečný. Ponúka možnosť zabránenia viac-cieľovým útokom. Jedná sa o útok na viacero používateľov daného kryptografického systému, resp. viacero kľúčov používateľa. Viac o tomto útoku je možné nájsť v \cite{mta}. V prípade použitia id parametra algoritmus rozšíri veľkosť kľúča. Následne pri vyhľadávaní kľúčov nedochádza k degradácií bezpečnosti a veľkosť tajného kľúča môže ostať 128 bitov. Týmto spôsobom bude zachovaná aj rovnaká bezpečnosť systému pred útokom. 
Príklad za účelom šifrovania správy P za pomoci tajného kľúča K s id, je na nasledujúci:
\begin{enumerate}
	\item{\lstinline|CYCLIST(K,id,E)|},
	\item{\lstinline|ABSORB(nounce)|},
	\item{C $\gets$ \lstinline|ENCRYPT(P)|}.
\end{enumerate}

\textbf{Tri spôsoby spracovania jednorázového kľúča} -- 3. parameter pri inicializácií režimu cyklista je \textit{counter}, resp. počítadlo. Jedná sa o dátový prvok vo forme bajtového reťazca, ktorý môže byť inkrementovaný. Spracúva sa po jednej číslici. Vďaka tomu sa obmedzuje počet informácií, ktoré vie útočník využiť pre rôzne vstupy. Pri inicializácií používateľ zvolí veľkosť počítadla v intervale $2 \leq B \leq 256$. Predpokladá sa, že počítadlo je reťazec z množiny $\mathbb{Z^*_B}$. Potom ak je počítadlo inicializované ako prázdny reťazec, tak množina všetkých možných hodnôt po inkrementácií je $\mathbb{Z^1_B}$. Pri každom ďalšom navýšení sa zvýši hodnota za $*$. Spracovanie prebieha od najvýznamnejšieho bitu. Čím menšia je hodnota B, tým menší je počet možných vstupov pri každej iterácií permutácie. Vďaka tomu je zabezpečená lepšia ochrana pred tzv. z ang \textit{Power analysis} \cite{paa} útokmi a jeho variantami. Za účelom zamedzenia týchto útokov sa používa \lstinline|ABSORB(nounce)| v prípade ak si počítadlo pri inicializácií zvolíme prázdny reťazec E. 

\subsubsection{Použitie za účelom autentizovaného šifrovania s bežným heslom}
% !TEX root = ../thesis.tex
\chapter{Konfigurácia prostredia, metodika testovania a merania}
\section{Prostredie virtuálnych strojov pomocou VirtualBox}\label{merania}
Pri testovaní funkcionality VPN sme použili virtualizačný nástroj (daľej \textbf{\acrshort{vm}}) Virtual Box (ďalej \textbf{\acrshort{vb})}, spoločnosti Oracle, vo verzii 6.1.30. \acrshort{vb} je voľne dostupný. Inštalácia je jednoduchá a rýchla. Viac informácii o nástroji je možné dohľadať v \cite{vbox}.

Pre použitie je potrebné aby mal používateľ k dispozícií obraz operačného systému (ďalej \acrshort{os}). Tie nie je problém získať ani pre \acrshort{os} Windows a podobne, avšak pri našej práci sme zvolili využitie voľno dostupného  \acrshort{os} -- \textbf{Linux Ubuntu} vo verziách 20.04.3(\acrshort{lts}\footnote{z ang. \acrlong{lts}})-- OSC a 21.10 -- OSS. Pri opise práce použijeme označenia OSS pre VPN server a OSC pre klienta.

Pri jednoduchej inštalácií \acrshort{vm} sme použili konfiguráciu s  2048 MB RAM a 2 jadrami. (Minimálna inštalácia). V prípade potreby dávame do popredia návod na prípravu Windows \acrshort{os} v \acrshort{vm} -- \cite{vmkonfig}, avšak postup je triviálny. Po inštalácii sme OS aktualizovali pomocou príkazov:
\begin{lstlisting}[language=bash]
	sudo apt-get update
	sudo apt-get upgrade
\end{lstlisting}
Následne sme doinštalovali potrebné súčasti k \acrshort{vm} \acrshort{os} vo verzii ako je \acrshort{vb}, teda 6.1.30. Dôvodom bolo zväčšenie rozlíšenia a využívanie možnosti zdieľaného priečinka s \acrshort{os}, na ktorom daný \acrshort{vb} beží. Za účelom správneho fungovania priečinka bolo nutné v termináli použiť príkaz:
\begin{lstlisting}[language=bash]
	sudo usermod -aG vboxsf $(whoami)
\end{lstlisting}
a následne reštartovať \acrshort{os}.

Posledné úpravy prostredia sú spojené s jazykom C a balíčkom Make. Inštalácia je opäť jednoduchá. Použili sme príkazy:
\begin{lstlisting}[language=bash]
	sudo apt install gcc
	sudo apt install make
\end{lstlisting}
Uvedené úpravy boli vykonané na oboch \acrshort{os}.

V prípade použitia \acrshort{os} Windows je postup inštalácie jazyka C a balička Make zložitejší. Používateľovi odporúčam použitie knižníc Winlibs, dostupne na \href{https://winlibs.com/}{webe}. Po stiahnutí balíčkov musí používateľ importovať uvedený balíček, resp. cestu k nemu, do premenných prostredia \acrshort{os} Windows. Jeden zo spôsobov je uvedený aj na Winlibs stránke. 

\section{Zmena sieťových adaptérov}
\acrshort{vb} ponúka rôzne možnosti nastavenia sieťových adaptérov. Aktuálne su dostupné tieto:
\begin{itemize}
	\item{Not attached}
	\item{Network Address Translation (ďalej NAT)}
	\item{NAT Network}
	\item{Bridge adapter}
	\item{Internal}
	\item{Host-Only}
	\item{Generic driver}
	\item{Cloud-Based} -- experimentálne
\end{itemize} 
 
Konektivita jednotlivých možností je znázornená pomocou obrázku \ref{vbmode}. Viac informácií o jednotlivých režimoch je dostupných na \cite{vboracle}.

\begin{figure}
	\centering
	\includegraphics[width=.9\textwidth]{figures/vbmodes}
	\caption{Konektivita jednotlivých sieťových adaptérov}
	\label{vbmode}
\end{figure}
Vzhľadom k našim potrebám, teda obojsmerná komunikácia medzi 2 VM, sú pre nás relevantné režimy bridge a NAT network. Ako si môžeme všimnúť, NAT vyžaduje dodatočnú konfiguráciu portov v prípadoch kedy chceme aby nastala komunikácia medzi dvoma VM. Z tohto dôvodu je pre čo najjednoduchší prístup zvoliť práve režim bridge. Ten priradí VM vlastnú IP adresu, pomocou, ktorej stroj komunikuje. 

Viac o jednotlivých režimov je taktiež možné nájsť v \cite{vbguide}. Autor sa venuje postupu konfigurácie jednotlivých režimov spoločne s ich opisom.

\section{Nástroje použité v obrazoch}
  winlibs, visual studio , gcc ,make, c kod na meranie poctu cyklov z bc + cas  
\chapter{Implementacia jednoduchej VPN siete}
https://learn.microsoft.com/en-us/windows/win32/winsock/finished-server-and-client-code
\section{Dead Simple VPN}\label{dsvpn}
Dead Simple VPN je voľne dostupný\footnote{\href{https://github.com/jedisct1/dsvpn}{https://github.com/jedisct1/dsvpn}} program, napísaný v jazyku C. Určený je pre operačný systém Linux. Autorom je Frank Denis. DSVPN rieši najbežnejší prípad použitia VPN, teda pripojenie klienta k VPN serveru cez nezabezpečenú sieť. Následne sa klient dostane na internet prostredníctvom servera. Uvedenú skutočnosť je možne vidieť na schéme \ref{vpnsimple}.
\begin{figure}
	\centering
	\includegraphics[width=0.9\textwidth]{figures/vpnsimple}
	\caption{Schéma jednoduchej VPN}
	\label{vpnsimple}
\end{figure}


DSVPN používa protokol riadenia prenosu -- \acrshort{tcp}\footnote{z ang. \textit{Transmission Control Protocol}} \cite{tcp}. Medzi ďalšie pozitíva patrí:
\begin{itemize}
	\item{Používa iba modernú kryptografiu s formálne overenými implementáciami,}
	\item{Malá a konštantná pamäťová stopa. Nevykonáva žiadne dynamické alokovanie pamäte (z ang. \textit{heap memory}),}
	\item{Malý (~25 KB) a čitateľný kód. Žiadne vonkajšie závislosti (z ang. \textit{Dependencies}),}
	\item{Funguje po preklade GCC prekladačom. Bez dlhej dokumentácia, žiaden konfiguračný súbor, dodatočná konfigurácia. DSVPN je spustiteľná jednoriadkovým príkazom na serveri, obdobne na klientovi. Bez potreby konfigurácie brány firewall a pravidiel smerovania,}
	\item{Funguje na Linuxe (kernel >= 3.17), macOS a OpenBSD, DragonFly BSD, FreeBSD a NetBSD v klientskych a point-to-point režimoch. Pridanie podpory pre iné operačné systémy je triviálne,}
	\item{Nedochádza k úniku IP medzi pripojeniami, ak sa sieť nezmení. Blokuje IPv6 na klientovi, aby sa zabránilo úniku IPv6 adries.}
\end{itemize} 

V uvedenej VPN autor zakomponoval aj možnosť pokročilejších nastavení. Celkový súhrn vstupných parametrov pri štarte programu je takýto:
\begin{lstlisting}[language=bash]
	./dsvpn   server
	<key file>
	<vpn server ip or name>|"auto"
	<vpn server port>|"auto"
	<tun interface>|"auto"
	<local tunnel ip>|"auto"
	<remote tunnel ip>"auto"
	<external ip>|"auto"

	./dsvpn   client
	<key file>
	<vpn server ip or name>
	<vpn server port>|"auto"
	<tun interface>|"auto"
	<local tunnel ip>|"auto"
	<remote tunnel ip>|"auto"
	<gateway ip>|"auto"
	\end{lstlisting} 
Väčšina parametrov je však prednastavených na automatické hodnoty. Príkladom je port 443, vytvorenie rozhrania tun0, prevziatie externej IP adresy zo siete a ďalšie. 

\section{Kryptografia použitá v DSVPN}
DSVPN používa v svojej implementácií malú sebestačnú kryptografickú knižnicu -- \textit{Charm}\footnote{\url{https://github.com/jedisct1/charm}}. Jej autorom je tvorca DSVPN. Implementácia umožňuje autentizované šifrovanie (z ang.\textit{authenticated encryption}) a hašovanie kľúčov (z ang. \textit{keyed hashing}). Správnosť implementácie algoritmu v knižnici programátor overil pomocou nástroja \textbf{Cryptol}\footnote{\url{https://cryptol.net/index.html}}. Uvedený nástroj slúži na zápis algoritmu do matematickej špecifikácií. Tým poskytne možnosť jednoduchšej a hlavne korektnej implementácie zvoleného kryptografického algoritmu. Zároveň je možné program využiť aj na verifikáciu vytvoreného riešenia. Obdobne sú v repozitári knižnice ponechané overovacie skripty pre jednoduché spustenie. 

Kryptografický algoritmus použitý v DSVPN je Xoodoo permutácia v duplex móde, pričom môže byť jednoducho nahradená napríklad Gimli-im\footnote{\url{https://github.com/jedisct1/gimli}} \cite{gimli} alebo Simpira384\footnote{\url{https://github.com/jedisct1/simpira384}} \cite{simpira}, ktorá je založené na AES-e . Uvedené algoritmy sú predmetom opisu kapitoly \ref{krypto}. Pri zmene musí používateľ zasiahnuť do zdrojového kódu v súbore \textbf{charm.c}, ktorého obsahom sú kryptografické primitíva.     
\section{Experimentálne overenie VPN}
DSVPN sme prakticky overili pomocou dvojice virtuálnych strojov OSS a OSC, ktorých opis je obsahom \ref{merania}. Na zariadení OSS sme pomocou balička make a GCC prekladača vykonali inštaláciu DSVPN. Obdobný postup je aplikovaný aj vo VM OSC. Na OSS spúšťame VPN Server, ktorý nám poskytne IP adresu, prostredníctvom ktorej budeme komunikovať s vonkajším svetom. Na spustenie a vytvorenie spojenie vykonáme nasledujúce úkony:
\begin{enumerate}
	\item Vygenerovanie zdieľaného kľúča:\begin{lstlisting}[language=bash]
		dd if=/dev/urandom of=vpn.key count=1 bs=32
	\end{lstlisting} 
-- zdieľaný kľúč, ktorý sme vygenerovali, sa nám uložil do súboru \textit{vpn.key}. Ten je potrebné vložiť do priečinka s programom \textit{dsvpn} v oboch zariadeniach -- OSS aj OSC. 
	\item OSS zariadenie: \begin{lstlisting}[language=bash] 
	sudo ./dsvpn server vpn.key auto 
	2340 auto 10.8.0.254 10.8.0.2
	\end{lstlisting} 
-- tento príkaz zabezpečí spustenie VPN servera na prostredí OSS s IP adresou, priradenou k vytvoren=emu tunelovaciemu rozhraniu s menom \textit{tun0}. Vytvorí sa pri spustení servera\footnote{Pomocou \lstinline|ip address show tun0| zistíme IPv4 adresu VPN servera.}. Príkazom ďalej definujeme portové číslo 2340, ktoré sa použije pri nastolení TCP spojenie medzi klientom a serverom. Poslednou konfiguráciou je priradenie IP adresy tunelov, ktoré bude využívať náš klient -- 10.8.0.254 a server -- 10.8.0.2. Používateľ má ešte možnosť nastaviť tzv. External IP. Tú by sme využili ak by sme spúšťali DSVPN na routri poskytovateľa internetu.  
	\item OSC zariadenie: \begin{lstlisting}[language=bash] 
	sudo ./dsvpn client vpn.key 192.168.88.62 
	2340 auto 10.8.0.2 10.8.0.254
	\end{lstlisting} 
-- uvedený príkaz zabezpečí, že sa pripojíme na VPN Server, ktorý ma ip adresu \textit{192.168.88.62} s portom 2340. Následne vzniká TCP spojenie. Dôležité je si všimnúť poradie adries tunelov. Je opačné ako v prípade servera. 
	\item V prípade úspešnej konektivity sa operácia podarila a pre okolitý svet sme viditelný pomocou IP adresy, ktorú sme zvolili. 
\end{enumerate}

Na overenie správnosti funkcionality nám postačí jednoduchý sieťový príkaz \lstinline|traceroute|. Napríklad \lstinline|traceroute google.sk|\footnote{vo Windows CMD prostredí: \lstinline|tracert google.sk|}. Prvá z uvedených adries je práve tá, ktorú dané zariadenie používa. 

V našom prípade bolo nutné použiť lokálne adresy vzhľadom na to, že obe VM bežia na jednom hosťovskom počítači. Obidva zariadenia sú tým pádom pripojené k jednému internetovému poskytovateľovi, čo má za následok takmer rovnaké smerovanie k vzdialenej doméne. 

Proces zistenia IP adresy VPN servera, po spustení, a overenie funkčnosti je následne znázornené pomocou \ref{ipu21},\ref{ipu20}, \ref{vpntru20}.
V  \ref{ipu21} sme žltou farbou znázornili IP adresu, na ktorej je VPN server dostupný. Oranžová farba znázorňuje IP adresy tunelu medzi serverom a klientom v tomto poradí. Následne v \ref{ipu20} môžeme vidieť to isté pre klienta.
  \begin{figure}
  	\centering
  	\includegraphics[width=0.9\textwidth]{figures/ipu21}
  	\caption{Zistenie IP adries VPN Servera na VM OSS}
  	\label{ipu21}
  \end{figure}
  

\begin{figure}
	\centering
	\includegraphics[width=0.9\textwidth]{figures/ipu20}
	\caption{Zistenie IP adries VPN Klienta na VM OSC}
	\label{ipu20}
\end{figure}
Nakoniec, v \ref{vpntru20} môžeme vidieť ako klient pri internetovej komunikácií používa namiesto svojej vlastnej, adresu poskytnutú VPN Serverom na zariadení OSS -- žltou zvýraznená IP. Červenou je zaškrnutá farba poskytovateľa internetu. 

\begin{figure}
	\centering
	\includegraphics[width=0.9\textwidth]{figures/vpntru20}
	\caption{Overenie funkcionality DSVPN pomocou traceroute}
	\label{vpntru20}
\end{figure}
\section{Princíp fungovania programu}
High level overview

\section{Analýza zdrojového kódu DSVPN}
 Pri analýze sa zameriame výhradne na dôležité časti kódu DSVPN pre programovací jazyk C. Repozitár pozostáva z jedného make-file balíčka\cite{make}. 3 hlavičkových (.h) a k ním korešpondujúcimi zdrojovými kódmi (.c), s pomenovaním:
 \begin{enumerate}
 	\item \textbf{charm.h} -- kryptografická knižnica so 6 funkciami,
 	\item \textbf{os.h} -- funkcie čítania a zápisu paketov, vytvorenia, aplikácie a zrušenia tunelu,
 	\item \textbf{vpn.h} -- deklarácia konštánt, endianity\cite{endianita} a niektorých závislostí OS.
 \end{enumerate}

 V spomenutých hlavičkových súboroch sa nachádzajú deklarácie funkcií, ktoré VPN používa. Definície sú obsahom .c súborov, ako je v jazyku C zaužívaným zvykom. Obsahom tejto podkapitoly je analýza týchto kódov. 
 
 \subsection{Súbor charm.h a charm.c}
 Obsahom sú prevažne funkcie slúžiace pri behu kryptografického algoritmu XOODOO. Charm.h pozostáva z 6 funkcií. Ich implementácia nie je až tak rozsiahla. Zaberá celkovo 337 riadkov. Jednotlivú implementáciu každej funkcie postupne opíšeme. 
 
   \begin{minipage}{\linewidth} 	
  	\begin{lstlisting}[frame=single,
  		numbers=left,
  		caption={Obsah charm.h}\label{charm.h},
  		basicstyle=\ttfamily\small, keywordstyle=\color{black}\bfseries,]
void uc_state_init(uint32_t st[12], const unsigned char key[32], 
				  			 	 const unsigned char iv[16]);
void uc_encrypt(uint32_t st[12], unsigned char *msg, 
							  size_t msg_len, unsigned char tag[16]);	
int uc_decrypt(uint32_t st[12], unsigned char *msg, 
			   			   size_t msg_len,
			   			   const unsigned char *expected_tag, 
			   			   size_t expected_tag_len);
void uc_hash(uint32_t st[12], unsigned char h[32],
			 			 const unsigned char *msg, size_t len);
void uc_memzero(void *buf, size_t len);
void uc_randombytes_buf(void *buf, size_t len);
  		 	\end{lstlisting}
  	\end{minipage}\\
 \subsection{Súbor os.h a os.c}
 Obsahom su funkcie, ktorých úlohami sú čítanie alebo pridanie \acrshort{gw}, vytvorenie a nastavenie tunelu v danom OS. Následne je aplikovaná úprava firewall pravidiel, tak aby všetka komunikácia bola presmerovaná na VPN server. Úprava firewall pravidiel je závislá od role, pod ktorou je VPN spustená. Teda či sa jedná o server alebo klienta. 
 Celkovo je obsahom 12 funkcií, ktoré riešia uvedené úlohy. Detail je možné vidiet v \ref{os.h}.
 
 \begin{minipage}{\linewidth} 	
	\begin{lstlisting}[frame=single,
		numbers=left,
		caption={Obsah ZK os.h}\label{os.h},
		basicstyle=\ttfamily\small, keywordstyle=\color{black}\bfseries,]
 ssize_t safe_read(const int fd, void *const buf_, size_t count, 
 						       const int timeout);
 ssize_t safe_write(const int fd, const void *const buf_, 
 							      size_t count,
 							      const int timeout);
 ssize_t safe_read_partial(const int fd, void *const buf_,
 							 					   const size_t max_count);
 ssize_t safe_write_partial(const int fd, void *const buf_, 
 							    		     	const size_t max_count);
 
 typedef struct Cmds {
 	const char *const *set;
 	const char *const *unset;
 } Cmds;
 
 Cmds firewall_rules_cmds(int is_server);
 int shell_cmd(const char *substs[][2], const char *args_str,
 			         int silent);
 const char *get_default_gw_ip(void);
 const char *get_default_ext_if_name(void);
 int tcp_opts(int fd);
 int tun_create(char if_name[IFNAMSIZ], const char *wanted_name);
 int tun_set_mtu(const char *if_name, int mtu);
 ssize_t tun_read(int fd, void *data, size_t size);
 ssize_t tun_write(int fd, const void *data, size_t size); 
\end{lstlisting}
\end{minipage}\\ 
 \subsection{Súbor vpn.h a vpn.c}
 Hlavičkový súbor obsahuje prevažne definovanie niektorých parametrov potrebných na správnu funkcionalitu VPN, spoločne s korekciou pre niektoré OS. Viď. \ref{vpn.h}. Ostatné parametre, ktoré bolí opísané pri spustení su definované práve v tomto súbore (MTU,Porty,IP atď.).
 
 \begin{minipage}{\linewidth} 	
 	\begin{lstlisting}[frame=single,
 		numbers=left,
 		caption={Obsah ZK vpn.h}\label{vpn.h},
 		basicstyle=\ttfamily\small, keywordstyle=\color{black}\bfseries,]
 /*UNIX-like OS Dependent Libraries*/
 #include <sys/ioctl.h>
 #include <sys/socket.h>
 #include <sys/types.h>
 #include <sys/uio.h>
 #include <sys/wait.h>
 #include <net/if.h>
 #include <netinet/in.h>
 #include <netinet/tcp.h>
 /*End UNIX-like OS Dependent Libraries*/
 /*OS setup dependencies*/
 #ifdef __linux__
 #include <linux/if_tun.h>
 #endif
 
 #ifdef __APPLE__
 #include <net/if_utun.h>
 #include <sys/kern_control.h>
 #include <sys/sys_domain.h>
 #endif
 
 #ifdef __NetBSD__
 #define DEFAULT_MTU 1500
 #else
 #define DEFAULT_MTU 9000
 #endif
 /*End of OS setup dependencies*/ 
 	\end{lstlisting}
\end{minipage}\\

\subsubsection{Main() -- Beh programu}
Pred samotným opisom by som rád upozornil na jeden fakt. Autor DSVPN používa vo veľkej miere zápis pomocou tzv. ternárnych operátorov. Viac informácií o tejto problematike je možné nájsť v \cite{ternary}.

Vpn.c je hlavným zdrojovým kódom DSVPN. V jeho vnútri nájdeme hlavnú, štartovaciu, funkciu main. Zároveň má prilinkované aj vyššie uvedené knižnice. Ako prvé dochádza k inicializácií premennej štruktúry \lstinline|Context|\ref{context}.

\begin{minipage}{\linewidth} 	
	\begin{lstlisting}[frame=single,
		numbers=left,
		caption={Štruktúra Context}\label{context},
		basicstyle=\ttfamily\small, keywordstyle=\color{black}\bfseries,]
typedef struct Context_ {
	const char *  wanted_if_name;
	const char *  local_tun_ip;
	const char *  remote_tun_ip;
	const char *  local_tun_ip6;
	const char *  remote_tun_ip6;
	const char *  server_ip_or_name;
	const char *  server_port;
	const char *  ext_if_name;
	const char *  wanted_ext_gw_ip;
	char          client_ip[NI_MAXHOST];
	char          ext_gw_ip[64];
	char          server_ip[64];
	char          if_name[IFNAMSIZ];
	int           is_server;
	int           tun_fd;
	int           client_fd;
	int           listen_fd;
	int           congestion;
	int           firewall_rules_set;
	Buf           client_buf;
	struct pollfd fds[3];
	uint32_t      uc_kx_st[12];
	uint32_t      uc_st[2][12];
} Context;   
 	\end{lstlisting}
\end{minipage}\\ 

 Následne dochádza k načítaniu kľúča pomocou pomocnej funkcie 
\\
 \lstinline|load_key_file()|\ref{load}. Úlohou je prepočítanie zdieľaného kľúča, pričom sa používa funkcia z os.c -- \lstinline|safe_read()|. Realizuje sa znakové spočítanie, v ktorom je zakomponovaná funkcia  \lstinline|poll()|\cite{poll}. V prípade zhody veľkosti, funkcia vracia 0.
 
 \begin{minipage}{\linewidth} 	
 	\begin{lstlisting}[frame=single,
 		numbers=left,
 		caption={Načítanie zdieľaného kľúča}\label{load},
 		basicstyle=\ttfamily\small, keywordstyle=\color{black}\bfseries,]
 static int load_key_file(Context *context, const char *file)
 {
 	unsigned char key[32];
 	int           fd;
 	
 	if ((fd = open(file, O_RDONLY)) == -1) {
 		return -1;
 	}
 	if (safe_read(fd, key, sizeof key, -1) != sizeof key) {
 		(void) close(fd);
 		return -1;
 	}
 	uc_state_init(context->uc_kx_st, key, 
 							 (const unsigned char *) "VPN Key Exchange");
 	uc_memzero(key, sizeof key);
 	
 	return close(fd);
 }
  	\end{lstlisting}
\end{minipage}\\ 

 Po preverení kľúča dochádza k priradeniu parametrov do štruktúry \lstinline|Context| na základe vstupu pri spustení DSVPN. V prípade že nedochádza k zmene GateWay (ďalej \acrshort{gw}) IP adresy, tak sa priradí pôvodná. Tá sa získa pomocou \\\lstinline|get_default_gw_ip()|, ktorá je deklarovaná v \ref{os.h}. Prostredníctvom shell príkazu \lstinline|ip route...| a \lstinline|read_from_shell_command()| funkcie, dochádza k extrakcii informácií priamo z príkazového riadka. Tento krok je teda závislý od \acrshort{os}, v ktorom používateľ pracuje. Nasleduje overenie návratových hodnôt z funkcií iba v prípade ak je DSVPN spustené ako Klient.
 
 Program v prípade servera pokračuje s \lstinline|get_default_ext_if_name()|. Obdobne ako v predchádzajúcom odstavci je realizácia funkcionality, vykonaná pomocou terminálu. Podstata spočíva v zistení mena rozhrania, ktoré bude presmerované na VPN server. 
 
 Po nastavení parametrov sa dostávame k vytvoreniu tunelovacieho rozhrania. V tomto kroku autor používa  funckiu \lstinline|tun_create|. Úloha je vysoko závislá od OS. Dôsledkom toho je možné vidieť vetvenie funkcionality vzhľadom k bežiacemu OS. Ako sa už spomínalo v úvode kapitoly. DSVPN poskytuje kompatibilitu pre 6 OS, medzi ktoré  patrí Linux, FreeBSD, NetBSD, OpenBSD, MacOS, DragonFly. Následne na základe systému dochádza k vytváraniu tunelu s prednastaveným, resp. zvoleným menom. Program ďalej nastaví hodnotu MTU.
 
 Po príprave tunelu ešte repetitívne dochádza k overeniu. Tento krok vykonáva funkcia \lstinline|resolve_ip|. Tá ma v sebe vnorené 2 funkcie, ktoré su menným ekvivalentom aj vo Windows knižniciach. Jedná sa o funkcie \lstinline|getaddrinfo| a \lstinline|getnameinfo|.  
 
 Posledným krokom súvisiacim s konfiguráciou prostredia je vytvorenie pravidla pre branu FireWall (ďalej \acrshort{fw}). Tento úkon realizuje \lstinline|firewall_rules()|. Tento proces je opäť systémovo závislý. Jeho realizácia je vykonaná pomocou globálne definovanej funkcie \lstinline|Cmds firewall_rules_cmds()|. Tá obsahuje súbor prednastavených príkazov. Ich úlohou je presmerovanie celej premávky cez novovzniknutý tunel.
 
 Posledným úkonom je samotný beh VPN. Ten spúšťa funkcia \lstinline|doit()|. 
 Doterajší beh je jednoducho znázornený pomocou flow diagramu \ref{fc1}.
 
\begin{figure}
	\centering
	\includegraphics[width=0.5\textwidth]{figures/fc1}
	\caption{Beh DSVPN}
	\label{fc1}
\end{figure}

 Od tohto momentu sa presúvame k postupnému vnáraniu do procesu spojenia a spracovania dát. Vo vnútri \lstinline|doit()| dochádza k vetveniu programu. V prípade, že je DSVPN spustená ako server spustí sa funkcia \lstinline|tcp_listener()|. V opačnom prípade \lstinline|client_reconnect()|. Ulohami funkcií je nastolenie TCP spojenia medzi clientom a serverom. Listener vytvára socket pomocou systemovej funkcie \lstinline|bind()|, ktorý následne čaká na klienta. Smerník na socket je uložený do štruktúry Context. V tejto vetve sa v Context ešte nastaví smerovanie na tento socket. \lstinline|Client_reconnect()| podľa prednastaveného počtu pokusov o znovu nadviazanie spojenia sa pokúša pomocou \lstinline|client_connect()| o spojenie. Pred týmto úkonom samozrejme dochádza k overeniu či už nedošlo k nadviazaniu spojenia a o to sa stará \lstinline|client_disconnect()|.


 
  \lstinline|Client_connect()| slúži na pripojenie klienta k serveru. Vykonáva úpravu pravidiel v \acrshort{fw}. Následne sa pokúša o nadviazania spojenia pomocou funkcie  \lstinline|tcp_client()|. Po tejto sérií úloh sa vraciame opäť do \lstinline|doit()|. Tu je vo  \lstinline|while| cykle vykonávaná funkcia \lstinline|event_loop()|, v ktorej dochádza k použitiu kryptografickej knižnice charm. Jej obsahom je inicializácia premenných. viď. \ref{el}. Schéma \ref{fc2} znázorňuje opísané skutočnosti.
  
  \begin{minipage}{\linewidth} 	
  	\begin{lstlisting}[frame=single,
  		numbers=left,
  		caption={Premenné funkcie event loop}\label{el},
  		basicstyle=\ttfamily\small, keywordstyle=\color{black}\bfseries,]
     struct pollfd *const fds = context->fds;
  Buf                  tun_buf;
  Buf *                client_buf = &context->client_buf;
  ssize_t              len;
  int                  found_fds;
  int                  new_client_fd;
    	\end{lstlisting}
\end{minipage}\\ 
 
 

\begin{figure}
	\centering
	\includegraphics[width=0.9\textwidth]{figures/fc2}
	\caption{Funkcia Doit()}
	\label{fc2}
\end{figure}

 \subsubsection{Funckia event\_loop()}
 Event\_loop je 111 riadkov dlhá funkcia. Jej obsah môžeme rozdeliť na overovací a výkonový. Úlohou niekoľkých if-ou vo funkcii je preverovanie signálov, spätných hodnôt a podobných premenných, ktoré by signalizovali chybu alebo používateľov záujem o ukončenie relácie programu. 
 
 Zaujímavé je taktiež predom definované makro \lstinline|BUFFERBLOAT_CONTROL|. Jeho úlohou je zamedzenie problému zvaného \textbf{Bufferbloat}. V skratke, je to nechcený jav, ktorý je zapríčinený nadmerným ukladaním paketov do vyrovnávacej pamäte, tzv. zahltenie. To ma za následok vysokú latenciu a tzv. Packet Delay Variation -- \acrshort{pdv} (Jitter), v paketovo-orientovaných sieťach. Viac o tejto problematike je možne si prečítať na \cite{bufferbloat}.
 
 Na druhej strane výkonové funkcie, ako hovorí ich názov. niečo vykonávajú. Do tejto kategórie sme zaradili funkcie:
 \begin{itemize}
 	\item\lstinline|tcp_accept()| -- slúži na nastolenie nového TCP spojenia s klientom, 
 	\item\lstinline|tun_read()| -- volá \lstinline|safe_read_partial| v prípade linuxového OS,
 	\item\lstinline|uc_encrypt()| -- kryptografické šifrovanie správy,
 	\item\lstinline|safe_write_partial()| -- používa štandardizovanú funkciu write vo while cykle, zapíše zašifrované dáta do buffera určeného pre odoslanie klientovi, vracia počet zapísaných dát,
 	\item\lstinline|safe_write()| -- používa sa v prípade ak došlo k zahlteniu paketmi,
 	\item\lstinline|client_reconnect()| -- slúži na obnovu spojenia v prípade chyby,
 	\item\lstinline|safe_read_partial()| -- obdobne ako pri write, používa read funkciu,
 	\item\lstinline|uc_decrypt()| -- kryptografické dešifrovanie správy,
 	\item\lstinline|tun_write()| -- volá \lstinline|safe_write| pri OS Linux.
 \end{itemize}
Metóda použitá pri zápise zašifrovaných dát vo funkcii \lstinline|event_loop()| je znázornená v \ref{ed}.

 \begin{minipage}{\linewidth} 	
 	\begin{lstlisting}[frame=single,
 		numbers=left,
 		caption={Spôsob zápisu šifrovaných dát}\label{ed},
 		basicstyle=\ttfamily\small, keywordstyle=\color{black}\bfseries,]
 		
 writenb = safe_write_partial(context->client_fd, tun_buf.len,
 			    	 		         		  2U + TAG_LEN + len); 
 if (writenb < (ssize_t) 0) {// kontrola zahltenia -- bufferbloat
 	context->congestion = 1; 
 	writenb             = (ssize_t) 0;
 }
// ak je iny objem ako maximum
 if (writenb != (ssize_t)(2U + TAG_LEN + len)) {
 	writenb = safe_write(context->client_fd, tun_buf.len + writenb,
 	2U + TAG_LEN + len - writenb, TIMEOUT); 
 }
   	\end{lstlisting}
\end{minipage}\\ 
 
\subsubsection{Šifrovanie a dešifrovanie}
Proces šifrovanie resp. dešifrovania správy nastáva na oboch stránách siete, teda pri klientovy aj serveri. \ref{SS} demonštruje šifrovanie implementované v funkcii \lstinline|uc_encrypt()|. Tento proces sme sa pokúsili opísať v grafe \ref{fc3}. 

\begin{figure}
	\centering
	\includegraphics[width=0.7\textwidth]{figures/fc3}
	\caption{Funkcia uc\_encrypt}
	\label{fc3}
\end{figure}

\begin{minipage}{\linewidth} 	
	\begin{lstlisting}[frame=single,
		numbers=left,
		caption={Šifrovanie správy}\label{SS},
		basicstyle=\ttfamily\small, keywordstyle=\color{black}\bfseries,]
 		
void uc_encrypt(uint32_t st[12], unsigned char *msg, 
					      size_t msg_len, unsigned char tag[16])
{	//spracovanie po 16 znakov
	unsigned char squeezed[16];
	unsigned char padded[16 + 1];
	size_t        off = 0;
	size_t        leftover;
	
	if (msg_len > 16) {
		for (; off < msg_len - 16; off += 16) {
			endian_swap_rate(st);
			memcpy(squeezed, st, 16);
			xor128(st, &msg[off]);
			endian_swap_rate(st);
			xor128(&msg[off], squeezed);
			permute(st);
		}
	}
	leftover = msg_len - off;
	memset(padded, 0, 16);
	mem_cpy(padded, &msg[off], leftover);
	padded[leftover] = 0x80;
	endian_swap_rate(st);
	memcpy(squeezed, st, 16);
	xor128(st, padded);
	endian_swap_rate(st);
	st[11] ^= (1UL << 24 | (uint32_t) leftover >> 4 << 25
				    | 1UL << 26); 
	xor128(padded, squeezed);
	mem_cpy(&msg[off], padded, leftover);
	permute(st);
	squeeze_permute(st, tag);
}
  	\end{lstlisting}
\end{minipage}\\ 

Ako je viditeľne v kóde dochádza k častému použitiu 2 funkcií. Nimi sú  \lstinline|xor128()| a  \lstinline|permute()|. Jedná sa o pomerne dôležité bloky pre správne fungovanie šifrovacieho algoritmu. Obsah prvej z uvedených je preto znázornený v \ref{xor}.

\begin{minipage}{\linewidth} 	
	\begin{lstlisting}[frame=single,
		numbers=left,
		caption={Funkcia xor128}\label{xor},
		basicstyle=\ttfamily\small, keywordstyle=\color{black}\bfseries,]
static inline void xor128(void *out, const void *in)
{
	#ifdef __SSSE3__
	_mm_storeu_si128((__m128i *) out,
	_mm_xor_si128(_mm_loadu_si128((const __m128i *) out),
	_mm_loadu_si128((const __m128i *) in)));
	#else
	unsigned char *      out_ = (unsigned char *) out;
	const unsigned char *in_  = (const unsigned char *) in;
	size_t               i;
	
	for (i = 0; i < 16; i++) {	//xororvanie jednotlivych znakov 
		out_[i] ^= in_[i];		//v 16 bitovom bloku spravy
	}
	#endif
}
  	\end{lstlisting}
\end{minipage}\\ 

Na druhej strane \lstinline|permute()| je pomerne rozsiahla funkcia pričom jej obsah sa rozprestiera na 100 riadkoch. Funkcionalita závisí od procesorových inštrukcií. Avšak nás bude zaujímať softvérová implementácia, ktorá je aj použitá pri behu. Dôvodom je, že naše zariadenie nemá k dispozícií uvedené procesorové inštrukcie. Blok, ktorý je použitý pri volaní sa nachádza v \ref{permute}.   

\begin{minipage}{\linewidth} 	
	\begin{lstlisting}[frame=single,
		numbers=left,
		caption={Funkcia Permute + makrá }\label{permute},
		basicstyle=\ttfamily\small, keywordstyle=\color{black}\bfseries,]
#define ROTR32(x, b) (uint32_t)(((x) >> (b)) | ((x) << (32 - (b))))
#define SWAP32(s, u, v)        \
do {                           \
	t      = (s)[u];             \
	(s)[u] = (s)[v], (s)[v] = t; \
} while (0)
	
static void permute(uint32_t st[12])
{
	uint32_t e[4], a, b, c, t, r, i;	
	for (r = 0; r < XOODOO_ROUNDS; r++) {
		for (i = 0; i < 4; i++) {
			e[i] = ROTR32(st[i] ^ st[i + 4] ^ st[i + 8], 18);
			e[i] ^= ROTR32(e[i], 9);
		}
		for (i = 0; i < 12; i++) {
			st[i] ^= e[(i - 1) & 3];
		}
		SWAP32(st, 7, 4);
		SWAP32(st, 7, 5);
		SWAP32(st, 7, 6);
		st[0] ^= RK[r];
		for (i = 0; i < 4; i++) {
			a         = st[i];
			b         = st[i + 4];
			c         = ROTR32(st[i + 8], 21);
			st[i + 8] = ROTR32((b & ~a) ^ c, 24);
			st[i + 4] = ROTR32((a & ~c) ^ b, 31);
			st[i] ^= c & ~b;
		}
		SWAP32(st, 8, 10);
		SWAP32(st, 9, 11);
	}	
}
	\end{lstlisting}
\end{minipage}\\ 
\section{Analýza výpočtových nárokov DSVPN}\label{analyza}
mozno sekcia nie kapitola, K VPN neviem co pozerat 
https://windowsreport.com/test-vpn/ \\
https://www.mdpi.com/1999-5903/14/9/264 \\
https://nordvpn.com/vpn-speed-test/ \\
https://ieeexplore.ieee.org/stamp/stamp.jsp?tp=\&arnumber=9142755 \\
https://citeseerx.ist.psu.edu/document?repid=rep1\&type=pdf\&doi=a5639b03df528315a8e901f8f6b2bad2824a643a \\
% !TEX root = ../thesis.tex
\chapter{Vyhodnotenie dosiahnutých výsledkov}
V rámci tejto kapitoly zhrnieme dosiahnuté výsledky z analýzy, experimentálnych meraní a rozšírenia kompatibility pre OS Windows.
\section{Analýza jednoduchej VPN siete}
V úvode praktickej časti sme postupnou analýzou prešli jednotlivé časti zdrojového kódu DSVPN. Názorne sme opisovali dôležité bloky a funkcionalitu implementovanú v tejto jednoduchej VPN. Na základe nadobudnutých znalostí sme vypracovali stavové diagramy, ktoré čitateľovi jednoznačne vysvetľujú čo sa v programe realizuje. Na základe týchto informácií, dokáže pochopiť ako sa v praxi realizuje VPN sieťové spojenie.
\section{Experimentálne meranie autentizovaného šifrovania pomocou permutácie XOODOO}
Popri analýze zdrojového kódu sme experimentálne overili funkcionalitu autentizovaného šifrovania pomocou permutácie XOODOO v jednoduchej VPN sieti.  Simuláciu bežného prostredia sme uskutočnili za pomoci 2 virtuálnych OS a~jedného natívneho OS, na ktorom virtuálizácia prebiehala. Virtualizované zariadenia mali obdobne aj obmedzený výpočtový výkon a pridelené prostriedky. To sa prejavovalo v pomalších odozvách systémov na požiadavky používateľa. 

Zo získaných výsledkov v tabuľke \ref{tabmer} sme usúdili, že virtuálne prostredie nie je vhodné pre meranie. Z toho dôvodu sme vytvorili program, ktorý používa kryptografické funkcie z implementácie DSVPN. XOODOO permutáciu sme inicializovali podľa vzoru v DSVPN. Následne sme vygenerovali náhodné dáta a tie sme zašifrovali a dešifrovali. Získané dáta sú znázornené v~tabuľke~\ref{tabmerlokal}. 

\begin{table}[h!]
	\centering
	\resizebox{\textwidth}{!}{%
		\begin{tabular}{c|c|c|c|c}
			\multirow{2}{*}{\bfseries Veľkosť dát [B]} &
			\multicolumn{2}{|c}{\bfseries uc\_encrypt()} &
			\multicolumn{2}{|c}{\bfseries uc\_decrypt()} 
			\\
			& Počet cyklov
			& Čas vykonávania [ms]
			& Počet cyklov
			& Čas vykonávania [ms] 
			\\\hline\hline
			52		 
			& 15 428
			& 0,005 600
			& 16 124
			& 0,005 700
			\\		
			250
			& 50 634
			& 0,017 600
			& 50 866
			& 0,017 700
			\\	
			500		
			& 97 817
			& 0,034 000
			& 97 469
			& 0,033 800
			\\
			1 000		
			& 189 312
			& 0,065 600
			& 189 167
			& 0,065 500
			\\ 
			1 500		
			& 280 691
			& 0,097 100
			& 280 749
			& 0,097 100
			\\
			3 000		
			& 557 322
			& 0,192 700
			& 557 960
			& 0,192 900
			\\
			4 500		
			& 834 997
			& 0,288 600
			& 835 374
			& 0,288 700
			\\
			6 000		
			& 1 110 120
			& 0,383 600
			& 1 109 279
			& 0,383 300
			\\
			7 500		
			& 1 397 974
			& 0,483 100
			& 1 386 258
			& 0,479 000
			\\
			9 000		
			& 1 769 841
			& 0,611 600
			& 1 661 352
			& 0,574 100			
		\end{tabular}%
	}
	\caption{Výsledky z experimentálnych meraní funkcií na šifrovanie a dešifrovanie v prostredí lokálneho zariadenia}
	\label{tabmerlokal}
\end{table} 
Z týchto výsledkov už je jednoznačne, že čas vykonávania šifrovania a dešifrovania, je priamo úmerný veľkosti dát vstupujúcich do XOODOO permutácie. To isté platí aj pre počet vykonaných cyklov. Rádovo sa čas vykonávania počas merania pohyboval pod úrovňou 1 milisekundy a to aj pri šifrovaní dát o veľkosti 9000 bajtov, ktorá predstavuje maximálnu možnú prenosovú veľkosť paketu (MTU). To znamená, že v implementácií DSVPN do (de)šifrovania nevstupujú väčšie dáta. Použitie permutácie XOODOO na zariadení je teda extrémne rýchle a jej použitie pri sieťovom prenose je z používateľského hľadiska zanedbateľné. 

Vytvorený zdrojový kód na meranie šifrovania a dešifrovania je obsahom Prílohy A.4. Konkrétne v priečinku \lstinline|MeasurementsInLocalEnvironment|.
    
\section{Výsledky dosiahnuté pri tvorbe kompatibility \\DSVPN pre OS Windows}
Jedným z cieľov našej práce bolo aj rozšírenie pôvodného riešenia DSVPN o~kompatibilitu pre OS Windows. Samozrejmosťou bolo zachovanie pôvodnej kompatibility v rámci ostatných OS. V princípe sa nám podarilo vytvoriť tunelové spojenie medzi VPN klientom a VPN serverom so zabezpečeným prenosom dát. Taktiež je možné nadviazať spojenie nie len medzi zariadeniami s OS Windows, ale vieme spojiť ľubovoľné OS z množiny podporovaných DSVPN.  

Problém riešenia nastáva pri smerovaní paketov naspäť k pôvodnému vlastníkovi. Pomocou sieťových príkazov a monitoringu premávky sa nám nepodarilo doteraz zistiť, prečo sa odpoveď na pôvodnú požiadavku nedokáže vrátiť na~zariadenie. Tento úkon sa deje aj napriek tomu, že pri monitoringu vidíme ako paket s odpoveďou opúšťa zariadenie a prichádza na to, ktoré má dostať odpoveď. Súčasná hypotéza je, že nám uniká niečo pri smerovaní paketu z pozície OS Windows. Tento problém sa budeme snažiť ešte v blízkej budúcnosti vyriešiť, nakoľko by sa nemalo jednať o veľký problém. Čitateľ bude mať prístup ku aktuálnemu kódu pomocou stránky github.   
% !TEX root = ../thesis.tex

\chapter{Záver}
\label{summary}
Cieľom našej práce bolo uvedenie čitateľa do problematiky VPN sietí, so zameraním na použitie kryptografického algoritmu XOODOO permutácie za účelom zabezpečenia komunikácie. V~prvej časti práci sme zadefinovali potrebné pojmy spojené s problematikou VPN sietí. Následne sme postupným opisom charakterizovali čo sa deje v~sieťach typu VPN. Na základe naštudovanej literatúry sme vytvorili klasifikácie VPN sietí. Rozdelenie bolo realizované podľa logickej topológie a dát vstupujúcich do~šifrovania. Pomocou nadobudnutých informácií sme charakterizovali a špecifikovali činnosti implementované v známych VPN protokoloch. V závere prvej kapitoly sme zhrnuli rozdelenie VPN sietí pomocou grafu.

Druhá kapitola sa venuje problematike ľahkej kryptografie. V nej sme charakterizovali tento pomerné nový odbor v oblasti kryptografie. Následne sme pokračovali s opisom XOODOO permutácie, ktorú zaraďujeme do tejto kategórie. XOODOO permutácia je súčasťou kryptografického balíka Xoodyak, ktorý je jedným z finalistov štandardizačného procesu NIST v ľahkej kryptografie. Opísali sme činnosti v algoritme a vysvetlili možnosti použitia. 

Tretia kapitola obsahuje postupy prípravy prostredí na experimentovanie. Konfiguráciu virtualizovaných OS a opis jednotlivých nástrojov, ktoré boli pri práci použité. Opísali sme Wintun adaptér na tvorbu virtuálnych tunelovacích rozhraní v OS Windows. Za účelom použitia pri výučbe v špecializovaných predmetoch orientovaných na bezpečnost v počítačových sieťach, sme vytvorili jednoduché demo, ktoré demonštruje odosielanie a prijímanie ICMP paketov pomocou tohto adaptéru.

Štvrtá kapitola sa zameriava na praktickú demonštráciu jednoduchej VPN siete s využitím XOODOO permutácie. V úvode sme charakterizovali program DSVPN. Následne sme praktickým experimentom overili jeho funkčnosť. Z tejto činnosti sme vypracovali článok do Zborníka vedeckých prác TUKE FEI \cite{clanok}. Podrobne sme analyzovali zdrojový kód DSVPN. Popri analýze sme názorným spôsobom ukázali čo sa deje s paketmi. Experimentálne sme zmerali rýchlosť a počet cyklov implementácie XOODOO permutácie v DSVPN. Súčasťou našej práce bolo aj vytvorenie kompatibility DSVPN o OS Windows a opis vykonaných zmien. 

V piatej kapitole sme zhrnuli dosiahnuté výsledky z praktickej časti práce. Celý obsah prílohy A vrátane samotnej práce sme uverejnili na git stránku Github pod profilom mr171hg\footnote{\url{https://github.com/mr171hg/DiplomaProject}}. Dôvodom je dostupnosť týchto informácií pre každého, kto sa potrebuje v problematike VPN zorientovať.  

Výsledná podoba práce spoločne s demonštratívnymi príkladmi zdrojových kódov a ostatnými prílohami boli spracované tak, aby bolo možné ich použitie v špecializovaných predmetoch zameraných na bezpečnosť v počítačových sieťach.

Pre účely rozšírenia práce by sme navrhovali pokračovať v práci na kompatibilite DSVPN. Rozšíriť implementáciu o viacero kryptografických algoritmov a následne experimentálne merať. Aplikovať program DSVPN na sieťové zariadenie a vyskúšať reálnu prevádzku. Refaktorovať a optimalizovať zdrojový kód. Rozšíriť program o možnosti pripojenia viacerých klientov na jeden server s využitím paralelného programovania s viac vláknami. 


% good linebraking of bibtex url
\setcounter{biburllcpenalty}{7000}
\setcounter{biburlucpenalty}{8000}

%% The bibliography
\printbibliography[heading=bibintoc]

\label{theend} % the last page of the thesis

% List of acronyms

% Glossaries
%\printglossary

%% Appendix
% !TEX root = ../thesis.tex

\chapter*{Zoznam príloh}
\addcontentsline{toc}{chapter}{Zoznam príloh}

\begin{description}
    \item[Príloha A] CD Médium -- viď. obsah CD média 
\end{description}


\appendix 
\renewcommand\chaptername{Príloha}
% !TEX root = ../thesis.tex

\chapter{Obsah CD Média} 

Obsah tohto média je dostupný na gite:
\begin{itemize}
	\item \url{https://github.com/mr171hg/DiplomaProject}
\end{itemize}

Pouzite obrazy, zdrojove kody...

% zivotopis autora
%\curriculumvitae\protect
%Táto časť\/ je nepovinná. Autor tu môže uviesť\/ svoje biografické
%údaje, údaje o~záujmoch, účasti na~projektoch, účasti na~súťažiach,
%získané ocenenia, zahraničné pobyty na~praxi, domácu prax, publikácie
%a~pod.

\end{document}
